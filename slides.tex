\documentclass{beamer}

\usetheme[coloredtitles, coloredblocks]{vub}
\usepackage{amsmath,amsfonts,amsthm,amssymb,mathtools}
% \usepackage[shortlabels]{enumitem}
% \usepackage{hyperref}
% \usepackage{cleveref}
\usepackage{tikz-cd}

% %% PACKAGES

\usepackage{amsmath,amsfonts,amsthm,amssymb,mathtools}
\usepackage[shortlabels]{enumitem}
\usepackage{hyperref}
\usepackage{cleveref}
\usepackage{tikz-cd}

%% Theorem environments
\theoremstyle{plain}
\newtheorem{theorem}{Theorem}
\newtheorem{lemma}[theorem]{Lemma}
\newtheorem{prop}[theorem]{Proposition}
\newtheorem{corollary}[theorem]{Corollary}
\newtheorem*{theorem*}{Theorem}
\newtheorem*{lemma*}{Lemma}
\newtheorem*{prop*}{Proposition}
\newtheorem*{corollary*}{Corollary}

\theoremstyle{definition}
\newtheorem{definition}[theorem]{Definition}
\newtheorem{example}[theorem]{Example}
\newtheorem{exercise}[theorem]{Exercise}
\newtheorem*{definition*}{Definition}
\newtheorem*{example*}{Example}
\newtheorem*{exercise*}{Exercise}

\theoremstyle{remark}
\newtheorem{remark}[theorem]{Remark}
\newtheorem*{remark*}{Remark}

%===========================================================%
%The code below customises theorem numbering
%===========================================================%

\theoremstyle{plain}
\newtheorem{innercustomgenericplain}{\customgenericname}
\providecommand{\customgenericname}{}
\newcommand{\newcustomtheoremplain}[2]{%
    \crefname{#2}{#2}{#2s}%
    \newenvironment{#1}[1]
    {%
        \renewcommand\customgenericname{#2}%
        \crefalias{innercustomgenericplain}{#2}%
        \renewcommand\theinnercustomgenericplain{##1}%
        \innercustomgenericplain
    }
    {\endinnercustomgenericplain}
}

\theoremstyle{definition}
\newtheorem{innercustomgenericdefinition}{\customgenericname}
\providecommand{\customgenericname}{}
\newcommand{\newcustomtheoremdefinition}[2]{%
    \crefname{#2}{#2}{#2s}%
    \newenvironment{#1}[1]
    {%
        \renewcommand\customgenericname{#2}%
        \crefalias{innercustomgenericdefinition}{#2}%
        \renewcommand\theinnercustomgenericdefinition{##1}%
        \innercustomgenericdefinition
    }
    {\endinnercustomgenericdefinition}
}

\theoremstyle{remark}
\newtheorem{innercustomgenericremark}{\customgenericname}
\providecommand{\customgenericname}{}
\newcommand{\newcustomtheoremremark}[2]{%
    \crefname{#2}{#2}{#2s}%
    \newenvironment{#1}[1]
    {%
        \renewcommand\customgenericname{#2}%
        \crefalias{innercustomgenericremark}{#2}%
        \renewcommand\theinnercustomgenericremark{##1}%
        \innercustomgenericremark
    }
    {\endinnercustomgenericremark}
}


\newcustomtheoremplain{customtheorem}{Theorem}
\newcustomtheoremplain{customlemma}{Lemma}
\newcustomtheoremplain{customprop}{Proposition}
\newcustomtheoremplain{customcorollary}{Corollary}

\newcustomtheoremdefinition{customdefinition}{Definition}
\newcustomtheoremdefinition{customexample}{Example}
\newcustomtheoremdefinition{customexercise}{Exercise}

\newcustomtheoremremark{customremark}{Remark}

\newcommand{\inv}{^{-1}}
\newcommand{\defeq}{\overset{\mathrm{def}}{=}}


\newcommand{\liff}{\leftrightarrow}
\newcommand{\lthen}{\rightarrow}
\newcommand{\opname}{\operatorname}
\newcommand{\surjto}{\twoheadrightarrow}
\newcommand{\injto}{\hookrightarrow}
\DeclareMathOperator{\img}{im} % Image
\DeclareMathOperator{\Img}{Im} % Image
\DeclareMathOperator{\coker}{coker} % Cokernel
\DeclareMathOperator{\Coker}{Coker} % Cokernel
\DeclareMathOperator{\Ker}{Ker} % Kernel
\DeclareMathOperator{\rank}{rank} % rank
\DeclareMathOperator{\Spec}{Spec} % spectrum
\DeclareMathOperator{\Tr}{Tr} % trace
\DeclareMathOperator{\pr}{pr} % projection
\DeclareMathOperator{\ext}{ext} % extension
\DeclareMathOperator{\pred}{pred} % predecessor
\DeclareMathOperator{\dom}{dom} % domain
\DeclareMathOperator{\cod}{cod} % codomain
\DeclareMathOperator{\ran}{ran} % range
\DeclareMathOperator{\Hom}{Hom} % homomorphism
\DeclareMathOperator{\Mor}{Mor} % morphisms
\DeclareMathOperator{\ob}{ob} % objects
\DeclareMathOperator{\mor}{mor} % morphisms
\DeclareMathOperator{\Fun}{Fun} % functors
\DeclareMathOperator{\Nat}{Nat} % natural transformations
\DeclareMathOperator{\End}{End} % endomorphism
\DeclareMathOperator{\Ann}{Ann} % annihilator

% Category Theory
\DeclareMathOperator{\Mod}{\mathbf{Mod}}
\DeclareMathOperator{\Top}{\mathbf{Top}}
\DeclareMathOperator{\Vect}{\mathbf{Vect}}
\DeclareMathOperator{\Ring}{\mathbf{Ring}}
\DeclareMathOperator{\Rng}{\mathbf{Rng}}
\DeclareMathOperator{\Ab}{\mathbf{Ab}}
\DeclareMathOperator{\Set}{\mathbf{Set}}
\DeclareMathOperator{\Sh}{\mathbf{Sh}}
\DeclareMathOperator{\PSh}{\mathbf{PSh}}

\newcommand{\ol}{\overline}
\newcommand{\ul}{\underline}
\newcommand{\wt}{\widetilde}
\newcommand{\wh}{\widehat}
\newcommand{\norm}[1]{\left\| #1 \right\|}
\newcommand{\inner}[2]{\left\langle #1 , #2 \right\rangle}


\input{definitions/letterfonts}

\newtheorem{proposition}{Proposition}

% \usepackage[width=400pt]{geometry}

\DeclareMathOperator{\Open}{\mathbf{Open}}
\DeclareMathOperator{\opp}{opp}
\DeclareMathOperator{\Cont}{Cont}
\DeclareMathOperator{\Cov}{Cov}
\DeclareMathOperator{\colim}{colim}
\DeclareMathOperator{\id}{id}

\title{A Fluid Introduction to Condensed Mathematics}
\author{Wannes Malfait}
\date{}

\begin{document}
\maketitle

\begin{frame}
    \frametitle{Motivation}

    \textbf{Problem}: Algebra and topology clash:
    \begin{equation*}
        \id \colon \bbR_{\text{discrete}} \to \bbR_{\text{Euclidean}}
    \end{equation*}
    is epi + mono, but not iso

    \medskip
    Solution?
    \pause

    $\leadsto$ Condensed abelian groups
\end{frame}

\begin{frame}
    \frametitle{Presheaves}

    \textbf{Idea}: Study better behaved objects.
    \pause

    \begin{definition}
        A \emph{presheaf} on a topological space $X$
        is a functor
        \begin{equation*}
            \mcF \colon \Open(X)^{\opp} \to \Set / \Rng / \Ab / \dots
        \end{equation*}
    \end{definition}
    Notation:
    \begin{itemize}
        \item Elements of $\mcF(U)$ are called \emph{sections}
        \item If $ i \colon U\injto V$ then $\mcF(i)$ is
              called the \emph{restriction} map:
              \begin{equation*}
                  s\in \mcF(V) \mapsto s|_U \in \mcF(U).
              \end{equation*}
    \end{itemize}

\end{frame}

\begin{frame}
    \frametitle{Sheaves}

    \begin{example}
        We always have the presheaf given by
        \begin{itemize}
            \item $\mcF(U) \mapsto Y$, and
            \item $\mcF(f) \mapsto \id_Y$
        \end{itemize}
    \end{example}
    \pause
    Problem:
    \begin{itemize}
        \item Want more structure
        \item Want global things to be defined locally
    \end{itemize}
    $\leadsto$ sheaf conditions:

\end{frame}

\begin{frame}
    \frametitle{Sheaf conditions}

    For an open cover $\{U_i\}_{i\in I}$ of $U$:
    \begin{itemize}
        \item \emph{uniqueness/locality}: If $s,t\in \mcF(U)$ are sections such that
              $s|_{U_i} = t|_{U_i}$ for all $i\in I$, then $s=t$.
              \pause
        \item \emph{gluing}: If $\{s_i \in \mcF(U_i)\}_{i\in I}$ is
              a collection of sections such that
              \begin{equation*}
                  s_i|_{U_i \cap U_j} = s_j|_{U_i \cap U_j} \text { for all } i, j,
              \end{equation*}
              then there is a section $s\in \mcF(U)$ such that $s_i = s|_{U_i}$ for all $i \in I$.
    \end{itemize}

\end{frame}
\begin{frame}
    \frametitle{Examples}

    Typical examples are rings of functions:
    \begin{itemize}
        \item $\Cont(-, \bbC)$
        \item $C^\infty(-, \bbC)$
    \end{itemize}
    \pause
    \medskip

    \medskip
    The constant functor $U \mapsto Y$ is \emph{not} a sheaf!

    Uniqueness axiom implies $\mcF(\emptyset) = \{*\}$
\end{frame}

\begin{frame}
    \frametitle{Limits and Colimits}

    \begin{proposition}
        Limits and colimits of presheaves can be computed
        section-wise, i.e. the functor $\mcF \mapsto \mcF(U)$
        commutes with all limits and colimits.
    \end{proposition}

    \pause
    \begin{example}
        $(\mcF\times \mcG )(U) = \mcF(U) \times \mcG(U)$
    \end{example}

\end{frame}

\begin{frame}
    \frametitle{Sheafification}

    \begin{center}
        Can we do the same for sheaves?
    \end{center}
    \pause
    \begin{itemize}
        \item Limits: ok.
        \item Colimits: not always a sheaf.
    \end{itemize}
    There is a solution:
    \pause
    \begin{theorem}
        The fully faithful inclusion $\iota \colon \Sh(X) \injto \PSh(X)$
        admits an \emph{exact} left adjoint: \emph{sheafification}.
    \end{theorem}
    \begin{example}
        Sheafification of $U \mapsto Y$, is $U \mapsto \Cont(U, Y_{\text{discrete}})$.
    \end{example}

\end{frame}

\begin{frame}
    \frametitle{Abelian Sheaves}

    \begin{proposition}
        The category of abelian sheaves on a topological space $X$
        is abelian, and has enough injectives.
    \end{proposition}

    \begin{example}
        If $M$ is an injective object in $\Ab$, i.e. a divisible group,
        then for $x\in X$,
        \begin{equation*}
            U \mapsto
            \begin{cases}
                M     & x \in U    \\
                \{*\} & x \notin U
            \end{cases}
        \end{equation*}
        is an injective sheaf.
    \end{example}

\end{frame}

\begin{frame}
    \frametitle{Sheaf Cohomology}

    If
    \begin{equation*}
        \begin{tikzcd}[ampersand replacement=\&]
            0 \rar \& \mcF \rar \& \mcF' \rar \& \mcF'' \rar \& 0
        \end{tikzcd}
    \end{equation*}
    is exact, then globally we only get
    \begin{equation*}
        \begin{tikzcd}[ampersand replacement=\&]
            0 \rar \& \mcF(X) \rar \& \mcF'(X) \rar \& \mcF''(X)
        \end{tikzcd}
    \end{equation*}
    \pause
    $\leadsto$ Right derived functors $H^i(X, -)$ of $\mcF \mapsto \mcF(X)$
    \begin{definition}
        The sheaf cohomology of $X$ with coefficients $A$ is $H^i(X, \Cont(- , A))$.
    \end{definition}

\end{frame}

\begin{frame}
    \frametitle{Computing Homology}

    We have
    \begin{equation*}
        \begin{tikzcd}[ampersand replacement=\&, sep=small, cramped]
            0 \rar \& \Cont(\bbZ, -) \rar \& \Cont(\bbR, -) \rar
            \& \Cont(\bbR/\bbZ, -) \rar \& 0
        \end{tikzcd}
    \end{equation*}
    since every map $U\subseteq S^1 \to S^1$ can be
    lifted \emph{locally} to a map $U \to \bbR$.

    \medskip
    But, there is no global lift of $\id\colon S^1 \to S^1$
    to $S^1 \to \bbR$.
    So $H^1(S^1, \bbZ) \neq 0$.

    \medskip
    \pause
    \begin{proposition}
        If $S$ is a CW-complex:
        \begin{equation*}
            H^i_{\text{sheaf}}(S, \bbZ) = H^i_{\text{singular}}(S, \bbZ).
        \end{equation*}
    \end{proposition}

\end{frame}

\begin{frame}
    \frametitle{Sites}

    \begin{itemize}
        \item Need a slight abstraction of sheaves on a space.
        \item Want sheaves on categories, not just spaces.
        \item Crucial ingredient: coverings.
    \end{itemize}
    \medskip

    \pause
    $\leadsto$ Definition of a site $\approx $ ``Category + notion of coverings''.

\end{frame}

\begin{frame}
    \frametitle{Condensed Sets}

    \begin{definition}
        A \emph{condensed set}$^*$ is a sheaf on the site of profinite sets
        with coverings given by jointly surjective maps.
    \end{definition}
    \medskip
    \pause
    % Concretely:
    % \begin{itemize}
    %     \item A functor $T \colon \{\text{profinite sets}\}^{\opp} \to \Set/\Rng/\Ab/...$
    %     \item $T(\emptyset) = \{*\}$.
    %     \item $T(S_1 \sqcup S_2) \to T(S_1) \times T(S_2)$ is a bijection.
    %     \item For $S' \surjto S$ a surjection, and $S' \times_S S'$ the pullback
    %           with projections $p_1, p_2$, the map
    %           \begin{equation*}
    %               T(S) \to \{x\in T(S') \mid p_1^*(x) = p_2^*(x)\in T(S'\times_S S')\}
    %           \end{equation*}
    %           is a bijection.
    % \end{itemize}

    For a topological space $X$, we have an associated
    condensed set $\underline{X} = \Cont(-, X)$.

\end{frame}
\begin{frame}
    \frametitle{Profinite sets}


    \begin{definition}
        A \emph{profinite set} is a compact, Hausdorff, totally disconnected space.
    \end{definition}
    \begin{example}
        A finite discrete space
    \end{example}
    \pause

    \begin{proposition}
        A limit of profinite sets is profinite.
        In particular, products of profinite sets
        are again profinite.
    \end{proposition}

\end{frame}

\begin{frame}
    \frametitle{The motivating problem}

    \textbf{Claim}: for condensed sets the map:
    \begin{equation*}
        \underline{\id} \colon \underline{\bbR_{\text{discrete}}} \to \underline{\bbR_{\text{Euclidean}}}
    \end{equation*}
    is no longer an epimorphism.
    \pause

    \medskip
    Let $S \subset \bbR_{\text{Euclidean}}$ be the cantor set.
    $ S \cong \prod_{n\in \bbN}\{0,1\}$,
    so $S$ is profinite.
    Hence:
    \begin{equation*}
        \Cont(S, \bbR_{\text{discrete}}) \subsetneq \Cont(S, \bbR_{\text{Euclidean}})
    \end{equation*}
\end{frame}

\begin{frame}
    \frametitle{Condensed cohomology}

    For a condensed set $T$, let $\bbZ[T]$ be the sheafification
    of $S \mapsto \bbZ[T(S)]$.
    \begin{definition}
        The cohomology of $S$ with coefficients $A$ is
        \begin{equation*}
            H^i (S, A) = \Ext^i(\bbZ[\underline{S}], \underline{A})
        \end{equation*}
    \end{definition}

    \pause
    \begin{theorem}[\cite{Sch2019LecturesCM}, Theorem 3.2]
        Let $S$ be a compact Hausdorff space. Then
        \begin{equation*}
            H^i_{\text{cond}}(S, A) \cong H^i_{\text{sheaf}}(S, A)
        \end{equation*}
    \end{theorem}
\end{frame}

\bibliographystyle{alpha}
\begin{frame}{References}
    \nocite{Sch2019LecturesCM}
    \nocite{Apa2021condensed}
    \nocite{stacks-project}
    \nocite{Sch2020MasterClass}
    \bibliography{references.bib}
\end{frame}
\end{document}
