\documentclass{article}

%% PACKAGES

\usepackage{amsmath,amsfonts,amsthm,amssymb,mathtools}
\usepackage[shortlabels]{enumitem}
\usepackage{hyperref}
\usepackage{cleveref}
\usepackage{tikz-cd}

%% Theorem environments
\theoremstyle{plain}
\newtheorem{theorem}{Theorem}[section]
\newtheorem{lemma}[theorem]{Lemma}
\newtheorem{prop}[theorem]{Proposition}
\newtheorem{corollary}[theorem]{Corollary}
\newtheorem*{theorem*}{Theorem}
\newtheorem*{lemma*}{Lemma}
\newtheorem*{prop*}{Proposition}
\newtheorem*{corollary*}{Corollary}

\theoremstyle{definition}
\newtheorem{definition}[theorem]{Definition}
\newtheorem{example}[theorem]{Example}
\newtheorem{exercise}[theorem]{Exercise}
\newtheorem*{definition*}{Definition}
\newtheorem*{example*}{Example}
\newtheorem*{exercise*}{Exercise}

\theoremstyle{remark}
\newtheorem{remark}[theorem]{Remark}
\newtheorem*{remark*}{Remark}

%===========================================================%
%The code below customises theorem numbering
%===========================================================%

\theoremstyle{plain}
\newtheorem{innercustomgenericplain}{\customgenericname}
\providecommand{\customgenericname}{}
\newcommand{\newcustomtheoremplain}[2]{%
    \crefname{#2}{#2}{#2s}%
    \newenvironment{#1}[1]
    {%
        \renewcommand\customgenericname{#2}%
        \crefalias{innercustomgenericplain}{#2}%
        \renewcommand\theinnercustomgenericplain{##1}%
        \innercustomgenericplain
    }
    {\endinnercustomgenericplain}
}

\theoremstyle{definition}
\newtheorem{innercustomgenericdefinition}{\customgenericname}
\providecommand{\customgenericname}{}
\newcommand{\newcustomtheoremdefinition}[2]{%
    \crefname{#2}{#2}{#2s}%
    \newenvironment{#1}[1]
    {%
        \renewcommand\customgenericname{#2}%
        \crefalias{innercustomgenericdefinition}{#2}%
        \renewcommand\theinnercustomgenericdefinition{##1}%
        \innercustomgenericdefinition
    }
    {\endinnercustomgenericdefinition}
}

\theoremstyle{remark}
\newtheorem{innercustomgenericremark}{\customgenericname}
\providecommand{\customgenericname}{}
\newcommand{\newcustomtheoremremark}[2]{%
    \crefname{#2}{#2}{#2s}%
    \newenvironment{#1}[1]
    {%
        \renewcommand\customgenericname{#2}%
        \crefalias{innercustomgenericremark}{#2}%
        \renewcommand\theinnercustomgenericremark{##1}%
        \innercustomgenericremark
    }
    {\endinnercustomgenericremark}
}


\newcustomtheoremplain{customtheorem}{Theorem}
\newcustomtheoremplain{customlemma}{Lemma}
\newcustomtheoremplain{customprop}{Proposition}
\newcustomtheoremplain{customcorollary}{Corollary}

\newcustomtheoremdefinition{customdefinition}{Definition}
\newcustomtheoremdefinition{customexample}{Example}
\newcustomtheoremdefinition{customexercise}{Exercise}

\newcustomtheoremremark{customremark}{Remark}

\newcommand{\inv}{^{-1}}
\newcommand{\defeq}{\overset{\mathrm{def}}{=}}


\newcommand{\liff}{\leftrightarrow}
\newcommand{\lthen}{\rightarrow}
\newcommand{\opname}{\operatorname}
\newcommand{\surjto}{\twoheadrightarrow}
\newcommand{\injto}{\hookrightarrow}
\DeclareMathOperator{\img}{im} % Image
\DeclareMathOperator{\Img}{Im} % Image
\DeclareMathOperator{\coker}{coker} % Cokernel
\DeclareMathOperator{\Coker}{Coker} % Cokernel
\DeclareMathOperator{\Ker}{Ker} % Kernel
\DeclareMathOperator{\rank}{rank} % rank
\DeclareMathOperator{\Spec}{Spec} % spectrum
\DeclareMathOperator{\Tr}{Tr} % trace
\DeclareMathOperator{\pr}{pr} % projection
\DeclareMathOperator{\ext}{ext} % extension
\DeclareMathOperator{\pred}{pred} % predecessor
\DeclareMathOperator{\dom}{dom} % domain
\DeclareMathOperator{\cod}{cod} % codomain
\DeclareMathOperator{\ran}{ran} % range
\DeclareMathOperator{\Hom}{Hom} % homomorphism
\DeclareMathOperator{\Mor}{Mor} % morphisms
\DeclareMathOperator{\ob}{ob} % objects
\DeclareMathOperator{\mor}{mor} % morphisms
\DeclareMathOperator{\Fun}{Fun} % functors
\DeclareMathOperator{\Nat}{Nat} % natural transformations
\DeclareMathOperator{\End}{End} % endomorphism
\DeclareMathOperator{\Ann}{Ann} % annihilator

% Category Theory
\DeclareMathOperator{\Mod}{\mathbf{Mod}}
\DeclareMathOperator{\Top}{\mathbf{Top}}
\DeclareMathOperator{\Vect}{\mathbf{Vect}}
\DeclareMathOperator{\Ring}{\mathbf{Ring}}
\DeclareMathOperator{\Rng}{\mathbf{Rng}}
\DeclareMathOperator{\Ab}{\mathbf{Ab}}
\DeclareMathOperator{\Set}{\mathbf{Set}}
\DeclareMathOperator{\Sh}{\mathbf{Sh}}
\DeclareMathOperator{\PSh}{\mathbf{PSh}}

\newcommand{\ol}{\overline}
\newcommand{\ul}{\underline}
\newcommand{\wt}{\widetilde}
\newcommand{\wh}{\widehat}
\newcommand{\norm}[1]{\left\| #1 \right\|}
\newcommand{\inner}[2]{\left\langle #1 , #2 \right\rangle}


% Things Lie
\newcommand{\kb}{\mathfrak b}
\newcommand{\kg}{\mathfrak g}
\newcommand{\kh}{\mathfrak h}
\newcommand{\kn}{\mathfrak n}
\newcommand{\ku}{\mathfrak u}
\newcommand{\kz}{\mathfrak z}
\DeclareMathOperator{\Ext}{Ext} % Ext functor
\DeclareMathOperator{\Tor}{Tor} % Tor functor
\newcommand{\gl}{\opname{\mathfrak{gl}}} % frak gl group
\renewcommand{\sl}{\opname{\mathfrak{sl}}} % frak sl group chktex 6

% More script letters etc.
\newcommand{\SA}{\mathcal A}
\newcommand{\SB}{\mathcal B}
\newcommand{\SC}{\mathcal C}
\newcommand{\SF}{\mathcal F}
\newcommand{\SG}{\mathcal G}
\newcommand{\SH}{\mathcal H}
\newcommand{\OO}{\mathcal O}

\newcommand{\SCA}{\mathscr A}
\newcommand{\SCB}{\mathscr B}
\newcommand{\SCC}{\mathscr C}
\newcommand{\SCD}{\mathscr D}
\newcommand{\SCE}{\mathscr E}
\newcommand{\SCF}{\mathscr F}
\newcommand{\SCG}{\mathscr G}
\newcommand{\SCH}{\mathscr H}

% Mathfrak primes
\newcommand{\km}{\mathfrak m}
\newcommand{\kp}{\mathfrak p}
\newcommand{\kq}{\mathfrak q}

% number sets
\newcommand{\RR}[1][]{\ensuremath{\ifstrempty{#1}{\mathbb{R}}{\mathbb{R}^{#1}}}}
\newcommand{\NN}[1][]{\ensuremath{\ifstrempty{#1}{\mathbb{N}}{\mathbb{N}^{#1}}}}
\newcommand{\ZZ}[1][]{\ensuremath{\ifstrempty{#1}{\mathbb{Z}}{\mathbb{Z}^{#1}}}}
\newcommand{\QQ}[1][]{\ensuremath{\ifstrempty{#1}{\mathbb{Q}}{\mathbb{Q}^{#1}}}}
\newcommand{\CC}[1][]{\ensuremath{\ifstrempty{#1}{\mathbb{C}}{\mathbb{C}^{#1}}}}
\newcommand{\PP}[1][]{\ensuremath{\ifstrempty{#1}{\mathbb{P}}{\mathbb{P}^{#1}}}}
\newcommand{\HH}[1][]{\ensuremath{\ifstrempty{#1}{\mathbb{H}}{\mathbb{H}^{#1}}}}
\newcommand{\FF}[1][]{\ensuremath{\ifstrempty{#1}{\mathbb{F}}{\mathbb{F}^{#1}}}}
% expected value
\newcommand{\EE}{\ensuremath{\mathbb{E}}}
\newcommand{\charin}{\text{ char }}
\DeclareMathOperator{\sign}{sign}
\DeclareMathOperator{\Aut}{Aut}
\DeclareMathOperator{\Inn}{Inn}
\DeclareMathOperator{\Syl}{Syl}
\DeclareMathOperator{\Gal}{Gal}
\DeclareMathOperator{\GL}{GL} % General linear group
\DeclareMathOperator{\SL}{SL} % Special linear group

%---------------------------------------
% BlackBoard Math Fonts :-
%---------------------------------------

%Captital Letters
\newcommand{\bbA}{\mathbb{A}}	\newcommand{\bbB}{\mathbb{B}}
\newcommand{\bbC}{\mathbb{C}}	\newcommand{\bbD}{\mathbb{D}}
\newcommand{\bbE}{\mathbb{E}}	\newcommand{\bbF}{\mathbb{F}}
\newcommand{\bbG}{\mathbb{G}}	\newcommand{\bbH}{\mathbb{H}}
\newcommand{\bbI}{\mathbb{I}}	\newcommand{\bbJ}{\mathbb{J}}
\newcommand{\bbK}{\mathbb{K}}	\newcommand{\bbL}{\mathbb{L}}
\newcommand{\bbM}{\mathbb{M}}	\newcommand{\bbN}{\mathbb{N}}
\newcommand{\bbO}{\mathbb{O}}	\newcommand{\bbP}{\mathbb{P}}
\newcommand{\bbQ}{\mathbb{Q}}	\newcommand{\bbR}{\mathbb{R}}
\newcommand{\bbS}{\mathbb{S}}	\newcommand{\bbT}{\mathbb{T}}
\newcommand{\bbU}{\mathbb{U}}	\newcommand{\bbV}{\mathbb{V}}
\newcommand{\bbW}{\mathbb{W}}	\newcommand{\bbX}{\mathbb{X}}
\newcommand{\bbY}{\mathbb{Y}}	\newcommand{\bbZ}{\mathbb{Z}}

%---------------------------------------
% MathCal Fonts :-
%---------------------------------------

%Captital Letters
\newcommand{\mcA}{\mathcal{A}}	\newcommand{\mcB}{\mathcal{B}}
\newcommand{\mcC}{\mathcal{C}}	\newcommand{\mcD}{\mathcal{D}}
\newcommand{\mcE}{\mathcal{E}}	\newcommand{\mcF}{\mathcal{F}}
\newcommand{\mcG}{\mathcal{G}}	\newcommand{\mcH}{\mathcal{H}}
\newcommand{\mcI}{\mathcal{I}}	\newcommand{\mcJ}{\mathcal{J}}
\newcommand{\mcK}{\mathcal{K}}	\newcommand{\mcL}{\mathcal{L}}
\newcommand{\mcM}{\mathcal{M}}	\newcommand{\mcN}{\mathcal{N}}
\newcommand{\mcO}{\mathcal{O}}	\newcommand{\mcP}{\mathcal{P}}
\newcommand{\mcQ}{\mathcal{Q}}	\newcommand{\mcR}{\mathcal{R}}
\newcommand{\mcS}{\mathcal{S}}	\newcommand{\mcT}{\mathcal{T}}
\newcommand{\mcU}{\mathcal{U}}	\newcommand{\mcV}{\mathcal{V}}
\newcommand{\mcW}{\mathcal{W}}	\newcommand{\mcX}{\mathcal{X}}
\newcommand{\mcY}{\mathcal{Y}}	\newcommand{\mcZ}{\mathcal{Z}}


%---------------------------------------
% Bold Math Fonts :-
%---------------------------------------

%Captital Letters
\newcommand{\bmA}{\boldsymbol{A}}	\newcommand{\bmB}{\boldsymbol{B}}
\newcommand{\bmC}{\boldsymbol{C}}	\newcommand{\bmD}{\boldsymbol{D}}
\newcommand{\bmE}{\boldsymbol{E}}	\newcommand{\bmF}{\boldsymbol{F}}
\newcommand{\bmG}{\boldsymbol{G}}	\newcommand{\bmH}{\boldsymbol{H}}
\newcommand{\bmI}{\boldsymbol{I}}	\newcommand{\bmJ}{\boldsymbol{J}}
\newcommand{\bmK}{\boldsymbol{K}}	\newcommand{\bmL}{\boldsymbol{L}}
\newcommand{\bmM}{\boldsymbol{M}}	\newcommand{\bmN}{\boldsymbol{N}}
\newcommand{\bmO}{\boldsymbol{O}}	\newcommand{\bmP}{\boldsymbol{P}}
\newcommand{\bmQ}{\boldsymbol{Q}}	\newcommand{\bmR}{\boldsymbol{R}}
\newcommand{\bmS}{\boldsymbol{S}}	\newcommand{\bmT}{\boldsymbol{T}}
\newcommand{\bmU}{\boldsymbol{U}}	\newcommand{\bmV}{\boldsymbol{V}}
\newcommand{\bmW}{\boldsymbol{W}}	\newcommand{\bmX}{\boldsymbol{X}}
\newcommand{\bmY}{\boldsymbol{Y}}	\newcommand{\bmZ}{\boldsymbol{Z}}
%Small Letters
\newcommand{\bma}{\boldsymbol{a}}	\newcommand{\bmb}{\boldsymbol{b}}
\newcommand{\bmc}{\boldsymbol{c}}	\newcommand{\bmd}{\boldsymbol{d}}
\newcommand{\bme}{\boldsymbol{e}}	\newcommand{\bmf}{\boldsymbol{f}}
\newcommand{\bmg}{\boldsymbol{g}}	\newcommand{\bmh}{\boldsymbol{h}}
\newcommand{\bmi}{\boldsymbol{i}}	\newcommand{\bmj}{\boldsymbol{j}}
\newcommand{\bmk}{\boldsymbol{k}}	\newcommand{\bml}{\boldsymbol{l}}
\newcommand{\bmm}{\boldsymbol{m}}	\newcommand{\bmn}{\boldsymbol{n}}
\newcommand{\bmo}{\boldsymbol{o}}	\newcommand{\bmp}{\boldsymbol{p}}
\newcommand{\bmq}{\boldsymbol{q}}	\newcommand{\bmr}{\boldsymbol{r}}
\newcommand{\bms}{\boldsymbol{s}}	\newcommand{\bmt}{\boldsymbol{t}}
\newcommand{\bmu}{\boldsymbol{u}}	\newcommand{\bmv}{\boldsymbol{v}}
\newcommand{\bmw}{\boldsymbol{w}}	\newcommand{\bmx}{\boldsymbol{x}}
\newcommand{\bmy}{\boldsymbol{y}}	\newcommand{\bmz}{\boldsymbol{z}}

%---------------------------------------
% Scr Math Fonts :-
%---------------------------------------

\newcommand{\sA}{{\mathscr{A}}}   \newcommand{\sB}{{\mathscr{B}}}
\newcommand{\sC}{{\mathscr{C}}}   \newcommand{\sD}{{\mathscr{D}}}
\newcommand{\sE}{{\mathscr{E}}}   \newcommand{\sF}{{\mathscr{F}}}
\newcommand{\sG}{{\mathscr{G}}}   \newcommand{\sH}{{\mathscr{H}}}
\newcommand{\sI}{{\mathscr{I}}}   \newcommand{\sJ}{{\mathscr{J}}}
\newcommand{\sK}{{\mathscr{K}}}   \newcommand{\sL}{{\mathscr{L}}}
\newcommand{\sM}{{\mathscr{M}}}   \newcommand{\sN}{{\mathscr{N}}}
\newcommand{\sO}{{\mathscr{O}}}   \newcommand{\sP}{{\mathscr{P}}}
\newcommand{\sQ}{{\mathscr{Q}}}   \newcommand{\sR}{{\mathscr{R}}}
\newcommand{\sS}{{\mathscr{S}}}   \newcommand{\sT}{{\mathscr{T}}}
\newcommand{\sU}{{\mathscr{U}}}   \newcommand{\sV}{{\mathscr{V}}}
\newcommand{\sW}{{\mathscr{W}}}   \newcommand{\sX}{{\mathscr{X}}}
\newcommand{\sY}{{\mathscr{Y}}}   \newcommand{\sZ}{{\mathscr{Z}}}


%---------------------------------------
% Math Fraktur Font
%---------------------------------------

%Captital Letters
\newcommand{\mfA}{\mathfrak{A}}	\newcommand{\mfB}{\mathfrak{B}}
\newcommand{\mfC}{\mathfrak{C}}	\newcommand{\mfD}{\mathfrak{D}}
\newcommand{\mfE}{\mathfrak{E}}	\newcommand{\mfF}{\mathfrak{F}}
\newcommand{\mfG}{\mathfrak{G}}	\newcommand{\mfH}{\mathfrak{H}}
\newcommand{\mfI}{\mathfrak{I}}	\newcommand{\mfJ}{\mathfrak{J}}
\newcommand{\mfK}{\mathfrak{K}}	\newcommand{\mfL}{\mathfrak{L}}
\newcommand{\mfM}{\mathfrak{M}}	\newcommand{\mfN}{\mathfrak{N}}
\newcommand{\mfO}{\mathfrak{O}}	\newcommand{\mfP}{\mathfrak{P}}
\newcommand{\mfQ}{\mathfrak{Q}}	\newcommand{\mfR}{\mathfrak{R}}
\newcommand{\mfS}{\mathfrak{S}}	\newcommand{\mfT}{\mathfrak{T}}
\newcommand{\mfU}{\mathfrak{U}}	\newcommand{\mfV}{\mathfrak{V}}
\newcommand{\mfW}{\mathfrak{W}}	\newcommand{\mfX}{\mathfrak{X}}
\newcommand{\mfY}{\mathfrak{Y}}	\newcommand{\mfZ}{\mathfrak{Z}}
%Small Letters
\newcommand{\mfa}{\mathfrak{a}}	\newcommand{\mfb}{\mathfrak{b}}
\newcommand{\mfc}{\mathfrak{c}}	\newcommand{\mfd}{\mathfrak{d}}
\newcommand{\mfe}{\mathfrak{e}}	\newcommand{\mff}{\mathfrak{f}}
\newcommand{\mfg}{\mathfrak{g}}	\newcommand{\mfh}{\mathfrak{h}}
\newcommand{\mfi}{\mathfrak{i}}	\newcommand{\mfj}{\mathfrak{j}}
\newcommand{\mfk}{\mathfrak{k}}	\newcommand{\mfl}{\mathfrak{l}}
\newcommand{\mfm}{\mathfrak{m}}	\newcommand{\mfn}{\mathfrak{n}}
\newcommand{\mfo}{\mathfrak{o}}	\newcommand{\mfp}{\mathfrak{p}}
\newcommand{\mfq}{\mathfrak{q}}	\newcommand{\mfr}{\mathfrak{r}}
\newcommand{\mfs}{\mathfrak{s}}	\newcommand{\mft}{\mathfrak{t}}
\newcommand{\mfu}{\mathfrak{u}}	\newcommand{\mfv}{\mathfrak{v}}
\newcommand{\mfw}{\mathfrak{w}}	\newcommand{\mfx}{\mathfrak{x}}
\newcommand{\mfy}{\mathfrak{y}}	\newcommand{\mfz}{\mathfrak{z}}


\usepackage[width=400pt]{geometry}

\DeclareMathOperator{\Open}{\mathbf{Open}}
\DeclareMathOperator{\opp}{opp}
\DeclareMathOperator{\Cont}{Cont}
\DeclareMathOperator{\Cov}{Cov}
\DeclareMathOperator{\colim}{colim}

\title{A Solid Exploration of Condensed Mathematics}
\author{Wannes Malfait}
\date{}

\begin{document}

\maketitle
\newpage
\tableofcontents

\section{Introduction}
Most of the contents of this paper will be based on the lecture
notes by Peter Scholze, who developed the theory of condensed mathematics
together with Dustin Clausen (\cite{Sch2019LecturesCM}).

\section{Sheaves}
A central role in this paper will be played by sheaves.
It therefore makes sense to study these objects more closely,
and attempt to gain some intuition before jumping straight into
the heart of the matter.
The first step, will be to convince our selves, that sheaves
are interesting enough to study, and that they form a nice type
of category. As such, we will dive into sheaf cohomology and some
applications of it in algebraic topology. But first, we must
define what a sheaf is.

\subsection{On Topological Spaces}
Consider the following situation in algebraic geometry.
We have some (affine) algebraic variety $X$, with the Zariski Topology.
Associated to $X$ is an ideal $I$ of all the polynomials that vanish
on $X$. Then, the quotient ring $\mcO_X(X) = k[x_1,\ldots,x_n]/I$, is
the coordinate ring of $X$. The elements of $\mcO_X(X)$ are the regular
functions on $X$. For an open $U\subseteq X$ we can also consider the
regular functions on $U$, which we will denote as $\mcO_X(U)$.
This mapping between open subsets of $X$ and rings,
given by $\mcO_X$, is precisely a sheaf.
In this case, $\mcO_X$ is known as the \emph{structure sheaf}.
In general, a sheaf will give some global data that can be defined
locally.
\begin{definition}[Presheaves]
    Let $X$ be a topological space.
    A \emph{presheaf of sets} $\mcF$ on $X$ consists of two things:
    \begin{enumerate}
        \item For each open $U\subseteq X$ a set $\mcF(U)$.
              These are known as the \emph{sections} of $\mcF$ over $U$.
        \item For each inclusion $U\subseteq V$ a map
              $\rho_{U,V}\colon \mcF(V) \to \mcF(U)$, such that $\rho_{U,U} = id_{\mcF(U)}$, and
              for $U\subseteq V \subseteq W$ we have $\rho_{U,V}\circ \rho_{V,W} = \rho_{U, W}$.
              We call the $\rho_{U,V}$ the \emph{restrictions}, and if $s\in \mcF(U)$,
              we will often denote $\rho_{V,U}(s) = s|_V$.
    \end{enumerate}
\end{definition}
\begin{remark}
    If we write $\Open(X)$ for the category of open sets of $X$,
    with morphisms given by the inclusion maps, then a presheaf is
    precisely a functor $\mcF\colon \Open(X)^{\opp} \to \Set$.
\end{remark}

This only defines what a ``presheaf'' is.
The sheaf condition will ensure that the value of $\mcF(U)$ can be
constructed by defining it locally on elements of a cover $\{U_i\}_{i\in I}$ of $U$.
It is exactly this interaction between local definitions and global objects
that makes sheaves so useful.
\begin{definition}[Sheaves]
    Let $X$ be a topological space, and $\mcF$ a presheaf of sets on $X$.
    We say that $\mcF$ is a sheaf, if it satisfies the following additional properties
    for any open $U\subseteq X$ and open cover $\{U_i\}_{i\in I}$ of $U$:
    \begin{itemize}
        \item (uniqueness/locality) If $s,t\in \mcF(U)$ are sections such that
              $s|_{U_i} = t|_{U_i}$ for all $i\in I$, then $s=t$.
        \item (gluing) If $\{s_i \in \mcF(U_i)\}_{i\in I}$ is
              a collection of sections such that
              \begin{equation*}
                  s_i|_{U_i \cap U_j} = s_j|_{U_i \cap U_j} \text { for all } i, j,
              \end{equation*}
              then there is a section $s\in \mcF(U)$ such that $s_i = s|_{U_i}$ for all $i \in I$.
    \end{itemize}
    In other words, if we have a bunch of sections
    that agree on overlaps, then we can uniquely glue them together.
\end{definition}
\begin{remark}
    \label{rem:cat_def_sheaf}
    We can reformulate the properties in a more categorical manner.
    Namely, for any open cover $\{U_i\}_{i\in I}$ of $U\subseteq X$,
    the following diagram:
    \begin{equation*}
        \begin{tikzcd}[cramped]
            \mcF(U) \rar
            &\prod\limits_{i\in I} \mcF(U_i) \rar[shift left] \rar[shift right]
            & \prod\limits_{i,j \in I} \mcF(U_i\cap U_j),
        \end{tikzcd}
    \end{equation*}
    has to be an equalizer. This diagram makes sense in any target category
    that has all limits. In terms of sets,
    the first map is given by $s \mapsto (s|_{U_i})_{i\in I}$. The pair of
    maps is given by the two possible restrictions, $(s_i)_{i\in I} \mapsto ((s_i|_{U_i \cap U_j})_{j\in I})_{i\in I}$
    and $(s_j)_{j\in I} \mapsto ((s_j|_{U_i \cap U_j})_{i\in I})_{j\in J}$.

    In this way, we can define sheaves on rings or abelian groups, or other
    categories with all (set-indexed) limits.
    For example, a presheaf of abelian groups on $X$, is a functor
    $\mcF \colon \Open(X)^{\opp} \to \Ab$.
\end{remark}

In many cases, defining some structure on sheaves, will come down
to defining it on $\mcF(U)$ in a compatible way.
As an example, we can look at morphisms of sheaves.
\begin{definition}[Morphisms of sheaves]
    Let $\mcF, \mcG$ be presheaves on a topological space $X$.
    A  \emph{morphism of presheaves} $\varphi \colon \mcF \to \mcG$ assigns to
    each open $U\subset X$ a morphism $\varphi_U \colon \mcF(U) \to \mcG(U)$
    compatible with the restriction maps, i.e. for $U\subseteq V \subseteq X$
    the following diagram commutes:
    \begin{equation*}
        \begin{tikzcd}
            \mcF(V) \rar["\varphi_V"] \dar["\rho^\mcF_{U,V}"]
            & \mcG(V) \dar["\rho^\mcG_{U,V}"] \\
            \mcF(U) \rar["\varphi_U"]
            & \mcG(U)
        \end{tikzcd}.
    \end{equation*}
    In other words, $\varphi$ is a natural transformation between $\mcF$ and $\mcG$.
    Using a bit of abusive notation, the compatibility can be read as
    $\varphi(s|_U) = \varphi(s)|_U$.
    A \emph{morphism of sheaves} is a morphism of the underlying presheaves.
\end{definition}

For a topological space $X$, we now have (pre)sheaves and morphisms between them.
These form a category. We will write $\PSh(X)$ for the category of presheaves of
sets on $X$, and $\Sh(X)$ for the category of sheaves of sets on $X$.
In the same way, we write $\Ab(X), \Ring(X), \Vect(X),\dots $ for the category of
sheaves of abelian groups, rings, vector spaces, \dots on $X$.
\begin{remark}
    We can "recover" the underlying space, by taking $X = \{*\}$,
    the one-point space. We have $\Sh(*) = \Set, \Ab(*) = \Ab, \dots$
\end{remark}

\subsubsection{Examples}
To solidify our understanding of sheaves, it will be beneficial to look
at some examples.
\begin{example}
    The simplest examples of sheaves are those were
    $\mcF(U)$ is not just a set, but a set of functions, and the
    restrictions correspond to actual restrictions.

\end{example}
\begin{example}
    Let $Y$ be another topological space, then we can define $\mcF(U) \defeq \Cont(U, Y)$
    where $\Cont(U,Y) = \{f\colon U \to Y \mid f \text{ continuous }\}$.
    The restrictions are defined as actual restrictions, and we get a presheaf. To see that
    it is a sheaf, we need to verify the gluing condition.
    Let $\{U_i\}$ be an open covering of $U\subseteq X$,
    and $f_i \colon U_i \to Y$ is a continuous map for each $i\in I$,
    such that $f_i|_{U_i \cap U_j} = f_j |_{U_i \cap U_j}$ for all $i,j\in I$.
    We can now define a map $f\colon U \to Y$ via $f(u) = f_i(u)$ for any $i\in I$
    such that $u\in U_i$. By assumption, this map is well-defined.
    To see that it is continuous, for any $V\subseteq Y$ open,
    we can write
    \begin{equation*}
        f\inv(V) = U \cap f\inv(V)=
        \bigcup_{i\in I} (U_i \cap f\inv(V)) =
        \bigcup_{i\in I}f_i\inv(V),
    \end{equation*}
    which is open since all the $f_i$ are continuous.

    So we see that $\mcF$ is a sheaf.
    In the case that $Y$ has the discrete topology,
    we call $\mcF$ the \emph{constant sheaf with value Y}.
    The sections of $\mcF$ over $U$ are the \emph{locally} constant functions,
    i.e. at each point $x\in U$ we can find an open neighborhood $V$ of $x$,
    such that $f$ is constant on $V$.
\end{example}
\begin{example}

    Let $f\colon X\to Y$ be a continuous map, then we get a sheaf
    on $X$ by the rule
    \begin{equation*}
        \Gamma(Y/X)(U) = \{s\colon U\to Y \mid f\circ s = 1_{U}\}.
    \end{equation*}
    The gluing construction is the same as in the previous example.
    To see that this yields another section, note that for
    any $u \in U$, $s(u) = s_i(u)$ for some $i\in I$, and hence
    $f(s(u)) = f(s_i(u)) = u$.
    We call $\Gamma(Y/X)$ the \emph{sheaf of sections of $f$}.
\end{example}
\begin{example}
    As we remarked above, the structure sheaf $\mcO_X$ is also a sheaf,
    where again, the restrictions are actual restrictions of functions.
    To show that the gluing of regular functions is again regular,
    requires some machinery from algebraic geometry, which falls outside
    the scope of this paper.
\end{example}

As it turns out, it is possible to generalize
the definition of sheaves to categories which
also have the notion of a covering. We will
need this notion when discussing condensed sets.
\subsection{On a Site}

The idea of a site was first introduced by
Alexander Grothendieck, and has proven to be
very useful in algebraic geometry.
For this subsection we largely follow
\cite[Part 1, Chapter 7]{stacks-project}.

\subsubsection{Coverings and Sites}

To define a site, we collect all the essential
properties that coverings of topological spaces have:
\begin{definition}[Coverings and Sites]
    A \emph{site} is a small category $\mcC$ together
    with a set $\Cov(\mcC)$ of \emph{coverings} of
    $\mcC$, where the following axioms hold:
    \footnote{We force $\Cov(\mcC)$ to be a set since we will take limits
        over all coverings. It is possible to let $\Cov(\mcC)$ be a class,
        and then show that it can be replaced with a set of coverings
        that gives rise to the same category of sheaves.
        See \cite[Remark 7.6.3]{stacks-project}.}
    \begin{enumerate}
        \item If $V \to U$ is an isomorphism, then $\{V \to U\}\in \Cov(\mcC)$.
        \item If $\{U_i \to U\}_{i\in I}$ is a covering, and for each $i\in I$,
              $\{V_{ij} \to U_i\}_{j \in J_i}$ is a covering as well, then so is the
              composition $\{V_{ij} \to U\}_{i\in I, j\in J_i}$.
        \item If $\{U_i \to U\}_{i\in I}$ is a covering and $V \to U$ is a morphism
              of $\mcC$, then the pullback $U_i\times_U V$ exists for all $i$
              and $\{U_i\times_U V \to V\}_{i\in I}$ is a covering.
    \end{enumerate}
\end{definition}

To make sense of this definition, let us look at the
canonical example.
\begin{example}
    Let $X$ be a topological space, then $\Open(X)$ is a
    site with coverings given by the open covers.
    Let us verify that the axioms hold:
    \begin{enumerate}
        \item The only isomorphisms in $\Open(X)$ are the identity maps.
              Since $\{U\}$ is an open cover of $U$ for any $U \in \Open(X)$,
              the first axiom is satisfied.
        \item If $U = \bigcup_{i\in I} U_i$ is an open cover of $U$,
              and for each $i \in I$, $U_i = \bigcup_{j\in J_i} U_{ij}$ is an
              open cover of $U_i$, then $U = \bigcup_{i\in I}\bigcup_{j \in J_i} U_{ij}$
              is an open cover of $U$.
        \item If $U = \bigcup_{i\in I} U_i$ is an open cover, and $V \subseteq U$, then
              $V = \bigcup_{i \in I} V \cap U_i$ is an open cover of $V$.
    \end{enumerate}
    Here we used that $U\times_W V = U\cap V$ for $U, V \subseteq W$,
    which follows from $S \subseteq V, S \subseteq U \implies S \subseteq V\cap U$.
\end{example}

The following example is still quite similar to the previous
example. The underlying site of a condensed set will be similar
to this site.
\begin{example}
    Let $G$ be a group, then we can consider the category
    of all $G$-sets, i.e. sets with a corresponding action
    by the group $G$. Morphisms are given by maps $f \colon X \to Y$
    satisfying $g\cdot f(x) = f(g\cdot x)$, i.e. $G$-equivariant
    maps. To make this into a site, we define the covers to be the
    families of jointly surjective maps. In other words
    $\{f_i \colon X_i \to X\}_{i\in I}$ is a cover if
    $\bigcup_{i\in I}f_i(X_i) = X$. Let us verify the axioms:
    \begin{enumerate}
        \item If $f \colon X \to Y$ is an isomorphism, then
              in particular it is surjective, so $\{f \colon X \to Y\}$ is a cover
              of $Y$
        \item If $\{f_i \colon X_i \to Y\}_{i\in I}$ is jointly surjective, and
              for each $i\in I$ the family $\{f_{ij} \colon X_{ij} \to X_j\}_{j\in J}$ is jointly
              surjective, then so is $\{f_i \circ f_{ij}\colon X_{ij} \to Y\}_{i\in I, j\in J}$.
        \item Let $\{f_i \colon X_i \to Y\}_{i\in I}$ be a cover,
              and $f\colon T \to Y$ a $G$-equivariant map. We claim that the pullback
              $X_i \times_Y T$ exists and is given by
              $S = \{(x,t)\in X_i \times T \mid f_i(x) = f(t)\}$.
              This is a $G$-set, as if $f_i(x) = f(t)$, then
              \begin{equation*}
                  f_i(g\cdot x) = g\cdot f_i(x) = g\cdot f(t) = f(g \cdot t).
              \end{equation*}
              Since $S$ is the pullback in $\Set$ and any $G$-equivariant map
              is in particular a map of sets, the claim follows.
              To see that the projection maps $\{X_i \times_Y T \to T\}$
              are jointly surjective, take $t\in T$. There exists $i\in I$
              such that $f(t) \in f_i(X_i)$, so $(x_i, t) \in X_i \times_Y T$
              for some $x_i \in X_i$.
    \end{enumerate}
\end{example}

In this new context, we have to reword the definitions
of presheaves and sheaves in a more abstract manner.
\begin{definition}[Presheaves and Sheaves on a Site]
    Let $\mcC$ be a small category. A \emph{presheaf on $\mcC$} is
    a contravariant functor $\mcF \colon \mcC \to \Set$.
    If $\mcC$ is a site, then a \emph{sheaf on $\mcC$} is a
    presheaf $\mcF$, such that for any covering $\{U_i \to U\}_{i\in I}$,
    the following diagram is an equalizer:
    \begin{equation*}
        \begin{tikzcd}
            \mcF(U) \rar["e"] &\prod\limits_{i\in I}\mcF(U_i)
            \rar[shift left, "p_0^*"] \rar[shift right, "p_1^*"']
            & \prod\limits_{i,j\in I}\mcF(U_i \times_U U_j)
        \end{tikzcd}.
    \end{equation*}
    The map $e$ is given by $s \mapsto (s|_{U_i})_{i\in I}$.
    For the second maps, take $i,j\in I$. Then we have projections
    $p^{(i,j)}_0 \colon U_i \times_U U_j \to U_i$ and
    $p^{(i,j)}_1 \colon U_i \times_U U_j \to U_j$.
    These result in maps
    $p^{(i,j), *}_0 \colon \mcF(U_i) \to \mcF(U_i \times_U U_j )$ and
    $p^{(i,j), *}_1 \colon \mcF(U_j) \to \mcF(U_i \times_U U_j )$.
    The maps $p_0^*$ and $p_1^*$ are
    then given at component $i,j$ by mapping $(s_k)_{k\in I}$ to
    $p^{(i,j), *}_0(s_i)$ and $p^{(i,j), *}_1(s_j)$ respectively.
\end{definition}

\subsubsection{Sheafification}
In a lot of constructions, the natural thing we do, will end up
being a presheaf, but not a sheaf in general. So, we will need some way to turn
presheaves into sheaves.
\begin{prop}
    \label{prop:sheafification}
    The fully faithful inclusion
    \begin{equation*}
        \iota \colon \Sh(\mcC) \injto \PSh(\mcC),
    \end{equation*}
    admits a left adjoint $\mcF \to \mcF^\sharp$, ``sheafification''.
\end{prop}

We will give an explicit construction of $\mcF^\sharp$.
As it turns out, ``sheafification'' is actually a two-step process.
The first step is making the presheaf separated, and then turning
the separated presheaf into a sheaf.
\begin{definition}
    We say that a presheaf $\mcF$ is \emph{separated} if
    $\mcF(U) \to \prod\limits_{i\in I} \mcF(U_i)$ is injective
    for any cover $\{U_i \to U\}_{i\in I} \in \Cov(\mcC)$.
\end{definition}
\begin{remark}
    Since any equalizer is a monomorphism, it follows from the definition
    of sheaves on a site, that a sheaf is separated.
\end{remark}
\begin{example}
    \label{exmp:bad_presheaves}
    The presheaf $\mcF$ on a topological space $X$, which maps every
    open to the same set $Y \neq \{*\}$ is not separated, as
    \begin{equation*}
        Y = \mcF(\emptyset) \to \prod_{i \in \emptyset} = \{*\},
    \end{equation*}
    is not injective if $Y \neq \{*\}$.

    On the other hand, the presheaf $\mcG$ on a topological space
    $X$ which maps every open $U$ to the constant functions $U \to Y$
    for some space $Y$, is separated but not a sheaf. After all,
    if $U = \cup_i U_i$ is an open cover of $U$, and $f,g \in \mcG(U)$
    are such that $f|_{U_i} = g|_{U_i}$ then $f = g$, since any $x\in U$
    is contained in some $U_i$, and hence $f(x) = f|_{U_i}(x) = g|_{U_i}(x) = g(x)$.
    Since there is only a single function $\emptyset \to Y$ there are no
    problems with empty coverings. $\mcG$ fails to be a sheaf
    in general, since if $U = U_1 \cup U_2$ with $U_1 \cap U_2 = \emptyset$
    and $U_1 \neq \emptyset \neq U_2$, then we can't ``glue'' the functions
    $f \colon U_1 \to Y, f(x) = y_1$ and $g\colon U_2 \to Y, f(x) =y_2$
    together to a constant function on $U$ if $y_1 \neq y_2$.
\end{example}
We first have a few lemmas regarding limits of presheaves and sheaves.
\begin{lemma}
    \label{lem:lims_colims_presheaves}
    Let $\mcC$ be a site, then limits and
    colimits exist in $\PSh(\mcC)$. Additionally, for any $U\in \mcC$,
    the functor $\PSh(\mcC) \to \Set \colon \mcF \mapsto \mcF(U)$
    commutes with all limits and colimits.
\end{lemma}
\begin{proof}
    Let $\mcF \colon \mcI \to \PSh(\mcC)$ be a diagram.
    We get a cone $(\mcF_{\lim}, p^i)$ by
    moving to the images in $\Set$. In other words we take
    $\mcF_{\lim} (U) = \lim_{i \in \mcI} \mcF_i(U)$,
    and $p^i_U \colon \lim_{i\in \mcI}\mcF_i(U) \to \mcF_i(U)$.
    If $U\to V$ is a morphism in $\mcC$, we get a unique map
    $\mcF_{\lim}(V) \to \mcF_{\lim}(U)$, given by the fact
    that $\mcF_{\lim}(U)$ is a limit, and $\mcF_{\lim}(V)$ is a cone
    on all the $\mcF_i(U)$ by composing the maps $\mcF_{\lim}(V) \to \mcF_i(V)$ with
    the maps $\mcF_i(V) \to \mcF_i(U)$. So we see that $\mcF_{\lim}$
    is indeed a presheaf.
    \begin{equation*}
        \begin{tikzcd}[column sep=small]
            &&\mcF_{\lim}(V) \dar[dashed] \ar[ddll, bend right, "p^i_V"'] \ar[ddrr, bend left, "p^{i'}_V"]\\
            &&\mcF_{\lim}(U) \ar[dl, "p^i_U"'] \ar[dr, "p^{i'}_U"]\\
            \mcF_i(V) \rar &\mcF_i(U) \ar[rr] &&\mcF_{i'}(U) &\mcF_{i'}(V) \lar
        \end{tikzcd}
    \end{equation*}
    To see that the maps $p^i$ are morphisms of presheaves, we need
    to verify that the following diagram commutes:
    \begin{equation*}
        \begin{tikzcd}
            \mcF_{\lim}(V) \dar \rar["p^i_V"] & \mcF_i(V) \dar \\
            \mcF_{\lim}(U) \rar["p^i_U"] & \mcF_i(U)
        \end{tikzcd},
    \end{equation*}
    but that already follows from the previous diagram.

    Let us now verify that $\mcF_{\lim}$ is actually a limit.
    If $(\mcG, g^i)$ is another cone, then for each $U\in \mcC$,
    we get a unique map $\mcG(U) \to \mcF_{\lim}(U)$, such that the
    corresponding cone diagrams commute. It suffices to show that these
    maps combine to form a map of presheaves.
    Using the universal property of limits, this comes down to showing
    \begin{equation*}
        \begin{tikzcd}
            \mcG(V)\rar \dar & \mcF_{\lim}(V) \dar \\
            \mcG(U)\rar & \mcF_{\lim}(U)
        \end{tikzcd}
        \iff
        \begin{tikzcd}
            \mcG(V)\rar["g^i_V"] \dar & \mcF_{i}(V) \dar \\
            \mcG(U)\rar["g^i_U"] & \mcF_{i}(U)
        \end{tikzcd}
        \forall i\in \mcI.
    \end{equation*}
    The commutativity of these last diagrams, is just saying that the
    maps $g^i$ are presheaf morphisms, which is true by $(\mcG, g^i)$ being a cone.
\end{proof}
The big difference between sheaves and presheaves, is that we can glue
things together defined on a cover. The trick will be to force
the presheaf to behave as we want. Let us try and make this more precise.

So far we have just talked about coverings as objects. We can
also consider maps between coverings. If $\mcU = \{U_i \to U\}_{i\in I}$
and $\mcV = \{V_j \to V\}_{j \in J}$ are two coverings, a morphism
of coverings between $\mcU$ and $\mcV$ is
a morphism $U \to V$ and a map $\alpha \colon I \to J$
together with morphisms $U_i \to V_{\alpha(i)}$
such that
\begin{equation*}
    \begin{tikzcd}
        U_i \ar[d] \rar & V_{\alpha(i)} \dar\\
        U \rar & V
    \end{tikzcd}
\end{equation*}
commutes. If $U = V$ and $U \to V$ is the identity,
we call $\mcU$ a \emph{refinement} of $\mcV$.
For a $ U \in \mcC$ the coverings together with refinements
gives a category $\Cov(U)$.
This allows the following reformulation:
\begin{lemma}
    \label{lem:presheaf_of_cover}
    Let $\mcF$ be a presheaf on a site $\mcC$.
    For $U \in \mcC$ and a cover $\mcU \in \Cov(U)$, define
    $\mcF(\mcU)$ as the following equalizer:
    \begin{equation*}
        \mcF(\mcU) =
        \{(s_i)_{i\in I} \in \prod\limits_{i\in I}\mcF(U_i)\mid
        s_i|_{U_i \times_U U_j} = s_j|_{U_i \times_U U_j}\}.
    \end{equation*}
    If $\mcU \to \mcV$ is a morphism of coverings, there is an induced
    map $\mcF(\mcV) \to \mcF(\mcU)$. This construction is functorial.
    Furthermore, since $\{1_U\}$ is a cover
    by the axioms of a site, this gives a map $\mcF(U) \cong \mcF(\{1_U\}) \to \mcF(\mcU)$.
    The presheaf $\mcF$ is a sheaf if and only if the map
    $\mcF(U) \to \mcF(\mcU)$ is bijective
    for every cover $\mcU$.
\end{lemma}
\begin{proof}
    Let $\mcU \to \mcV$ be a map of coverings. We have the following
    commutative diagram:
    \begin{equation*}
        \begin{tikzcd}
            & U_i \rar \dar &V_{\alpha(i)} \dar \\
            U_i \times_U U_j \ar[ru] \ar[rd]
            & U \rar & V
            & V_{\alpha(i)} \times_V V_{\alpha(j)} \ar[lu] \ar [ld] \\
            & U_j \rar \uar & V_{\alpha(j)} \uar
        \end{tikzcd},
    \end{equation*}
    which gives a unique map $U_i \times_U U_j \to V_{\alpha(i)} \times_V V_{\alpha(j)}$
    making the diagram commute. Applying $\mcF$ gives the diagram:

    \begin{equation}\label{eq:fibre_product_coverings}
        \begin{tikzcd}
            \mcF(U_i) \dar &\mcF(V_{\alpha(i)}) \lar \dar \\
            \mcF(U_i \times_U U_j) & \mcF(V_{\alpha(i)} \times_V V_{\alpha(j)} )  \lar \\
            \mcF(U_j) \uar &\mcF(V_{\alpha(j)}) \lar \uar
        \end{tikzcd}
        .
    \end{equation}

    We can now define $\mcF(\mcV) \to \mcF(\mcU)$ by
    \begin{equation*}
        (s_j)_{j\in J} \mapsto ((s_{\alpha(i)})|_{U_i})_{i\in I}.
    \end{equation*}
    That this map is well-defined follows from \cref{eq:fibre_product_coverings}.
    Indeed, if $(s_j)_{j\in J} \in \mcF(\mcV)$, then
    \begin{equation*}
        (s_{\alpha(j)}|_{V_{\alpha(i)}\times_U V_{\alpha(j)}})|_{U_i \times U_j}
        = (s_{\alpha(i)}|_{V_{\alpha(i)}\times_U V_{\alpha(j)}})|_{U_i \times U_j},
    \end{equation*}
    and hence
    \begin{equation*}
        (s_{\alpha(i)}|_{U_i})|_{U_i \times_U U_j}
        = (s_{\alpha(j)}|_{U_j})|_{U_i \times_U U_j}.
    \end{equation*}

    It remains to show that this construction is functorial, since
    the final claim is just the definition of a sheaf of sites. Let,
    to this purpose, $\mcU \to \mcV \to \mcW$ be maps of coverings,
    with associated maps $\alpha\colon I \to J$ and $\beta \colon J \to K$.
    The map $\mcF(\mcW) \to \mcF(\mcU)$ maps $(s_k)_{k\in K}$ to
    $(s_{\beta(\alpha(i))}|_{U_i})_{i\in I}$. The other map is given by
    first sending $(s_k)_{k \in K}$ to $(s_{\beta(j)}|_{V_j})_{j\in J}$,
    which is then sent to $((s_{\beta(\alpha(i))}|_{V_{\alpha(i)}})|_{U_i})_{i\in I}$.
    The result now follows by functoriality of $\mcF$.
\end{proof}
We now have the following construction:
\begin{lemma}
    Given a presheaf of $\mcF$ on a site $\mcC$, we can construct
    a new presheaf $\mcF^+$ by setting
    \begin{equation*}
        \mcF^+(U) = \colim_{\mcU \in \Cov(U)} \mcF(\mcU),
    \end{equation*}
\end{lemma}
\begin{proof}
    Let us first show how $\mcF^+$ acts on morphisms. Consider
    a morphism $U \to V$. Then this gives a map $\Cov(V) \to \Cov(U)$
    by mapping a cover $\{V_i \to V\}_{i\in I}$ of $V$ to the cover
    $\{V_i \times_V U \to U\}_{i\in I}$ of $U$,
    which exists by the third axiom of coverings.
    From the previous lemma there are induced maps
    ${\mcF}(\mcU) \to {\mcF}(\mcV)$.
    Since we have maps $\mcF(\mcV) \to \mcF^+(V)$,
    we obtain maps ${\mcF}(\mcU) \to \mcF^+(V)$.
    By the universal property of the colimit, this gives a unique map
    $\mcF^+(V) \to \mcF^+(U)$, making the cone diagrams commute.
    \begin{equation*}
        \begin{tikzcd}[column sep=small]
            &&\mcF^+(U) \ar[ddll, bend right, leftarrow]
            \ar[ddrr, bend left, leftarrow]\\
            &&\mcF^+(V)\uar[dashed]\\
            \mcF(\mcU) \rar &\mcF(\mcV)\ar[ur] \ar[rr]
            &&\mcF(\mcV') \ar[ul] &\mcF(\mcU') \lar
        \end{tikzcd}
    \end{equation*}

    It remains to show that this defines a functor $\mcC^{\opp}\to \Set$.
    If we have morphisms $U\to V \to W$, then
    \begin{equation*}
        W_i \times_W U \cong (W_i \times_W V) \times_V U,
    \end{equation*}
    since both are pullbacks of $\begin{tikzcd}
            [sep=small, cramped]
            W_i \rar & W & U \lar
        \end{tikzcd}$.
    So $\Cov(-)$ is a functor, and
    since $\mcU \mapsto \mcF(\mcU)$ is a functor
    by the previous lemma, the result now follows.
\end{proof}
It is possible to be more explicit about how $\mcF^+$ looks like.
For two coverings $\mcU, \mcU'$ of $U \in \mcC$,
we have a common refinement $\{U_i \times_U U'_j \to U\}_{i\in I, j\in J}$
which exists by the second and third axioms.
Furthermore, one can show that if $f,g\colon \mcU \to \mcV$
are refinements, then $\mcF(f) = \mcF(g)$ (\cite[Lemma 7.10.6]{stacks-project}).
This gives that $\Cov(U)^{\opp} \to \Set$ is a filtered
diagram. So
\begin{equation*}
    \mcF^+(U) =\Bigl( \coprod\limits_{\mcU \in \Cov(U)} \mcF(\mcU)\Bigr)/ \sim,
\end{equation*}
Where $s \sim s'$ if and only if there are covers $\mcU, \mcU'$ with
$s\in \mcF(\mcU), s' \in \mcF(\mcU')$ and a common refinement $\mcV$ such that
\begin{equation*}
    s_{\alpha(i)}|_{V_i} = s'_{\beta(i)}|_{V_i}, \, \forall i\in I.
\end{equation*}
We now come to the main theorem, from which
\cref{prop:sheafification} will follow.
\begin{theorem}
    Let $\mcF$ be a presheaf on a site. Then
    \begin{enumerate}
        \item The presheaf $\mcF^+$ is separated.
        \item If $\mcF$ is separated, then $\mcF^+$ is a sheaf.
        \item If $\mcF$ is a sheaf, then $\mcF \to \mcF^+$ is an isomorphism.
    \end{enumerate}
\end{theorem}
\begin{proof}
    \leavevmode
    \begin{enumerate}
        \item We need to show that $s \mapsto (s|_{U_i})_{i\in I}$ is injective
              for any cover $\{U_i \to U\}_{i \in I} \in \Cov(\mcC)$.
              Take $s, s' \in \mcF^+(U)$ such that
              $(s|_{U_i})_{i\in I} = (s'|_{U_i})_{i\in I}$ for some
              cover $\mcU = \{U_i \to U\}_{i\in I}$.
              By the description above of $\mcF^+$, we know that
              we can find covers $\mcV$ and $\mcV'$ of $U$,
              such that $s\in \mcF(\mcV)/\sim$ and $s'\in\mcF (\mcV')/\sim$.
              Let $\mcW$ be a common refinement of the three covers
              $\mcU, \mcV, \mcV'$. Since it is a refinement of $\mcU$,
              we have that
              \begin{equation*}
                  s'|_{W_j} = (s|_{U_{\alpha(j)}})|_{W_j} =
                  (s|_{U_{\alpha(j)}})|_{W_j} = s'|_{W_j},
              \end{equation*}
              so that $s=s'$.
        \item We need to verify the sheaf condition. Let $\{U_i \to U\}_{i\in I}$
              be a cover of $U$, and for each $i\in I$, $s_i \in \mcF^+(U_i)$ such that
              $s_i |_{U_i \times_U U_j} = s_j |_ {U_i \times_U U_j}$ for every $i,j \in I$.
              Since $\mcF$ is separated, the map $s \to (s|{U_i})_{i\in I}$ is injective.
              It is hence enough to show that there is some $s\in \mcF^+(U)$ with
              $s|_{U_i} = s_i$ for all $i\in I$.
              For each $i\in I$ we can find a cover $\mcU_i = \{U_{ij} \to U_i\}$
              such that $s_i \in \mcF(\mcU_i)/\sim$, and hence $s_{ij} \in \mcF(U_{ij})$
              such that $s_i |_{U_{ij}} = s_{ij}/\sim$. In the same way
              as \cref{lem:presheaf_of_cover} we get that
              \begin{equation*}
                  s_{ij}|_{U_{ij}\times_U U_{i'j'}} = s_{i'j'}|_{U_{ij}\times_U U_{i'j'}}.
              \end{equation*}
              So, $(s_{ij})_{i,j\in I} \in \mcF(\{U_{ij}\to U\}_{i,j\in I})$, and we can
              take $s = (s_{ij})_{i,j\in I}/\sim$.
              We just need to verify that $s|_{U_i} = s_i$. This follows
              from $(s|_{U_i})|_{U_{ij}} = s|_{U_{ij}} = s_i|_{U_{ij}}$, as $\mcF^+$ is
              also separated.
        \item This follows from \cref{lem:presheaf_of_cover}.
    \end{enumerate}
\end{proof}

With this we are now ready to prove \cref{prop:sheafification}.
We define $\mcF^\sharp = \mcF^{++}$.
\begin{proof}[Proof of \cref{prop:sheafification}]
    We first note that for any map of presheaves $\alpha\colon \mcF \to \mcG$,
    we have a commutative diagram
    \begin{equation*}
        \begin{tikzcd}
            \mcF \rar \dar["\alpha"] & \mcF^+ \dar["\alpha^+"] \\
            \mcG \rar & \mcG^+
        \end{tikzcd},
    \end{equation*}
    where $\alpha^+_U(s/ \sim) = \alpha_U(s)/\sim$.
    That this commutes, is by construction of $\alpha^+$.
    Using this, we can now show that
    \begin{equation*}
        \Hom_{\Sh(\mcC)}(\mcF^\sharp, \mcG) = \Hom_{\PSh(\mcC)}(\mcF, \iota(\mcG)).
    \end{equation*}
    The bottom row of the diagram
    \begin{equation*}
        \begin{tikzcd}
            \mcF \rar \dar &\mcF^+ \dar \rar & \mcF^{++} = \mcF^{\sharp} \dar \\
            \iota(\mcG) \rar  &\iota(\mcG)^+  \rar & \iota(\mcG)^{++} = \mcG^{\sharp}
        \end{tikzcd}
        ,
    \end{equation*}
    consists of isomorphisms by the previous theorem.
    So, any map $\mcF \to \iota(\mcG)$ gives rise to a map
    $\mcF^\sharp \to \mcG$.
    Conversely, since every $s\in \mcF^\sharp(U)$ comes from sections
    in $\mcF(U)$, we can lift any map $\mcF^\sharp \to \mcG$
    to a map $\mcF \to \mcG$.
\end{proof}

As an immediate consequence, we get that $(-)^\sharp$
commutes with all colimits. So $\Sh(\mcC)$ has all colimits,
since $\PSh(\mcC)$ has all colimits by \cref{lem:lims_colims_presheaves}.
We can say more:
\begin{corollary}
    The functor $(-)^\sharp\colon \PSh(\mcC) \to \Sh(\mcC)$ is exact.
\end{corollary}
\begin{proof}
    Since it is a left adjoint, it is right exact. On the other hand,
    colimits over filtered diagrams commute with finite limits.
    So, $(-)^\sharp$ is left exact as functor between presheaves.
    We claim that if $\mcI \to \Sh(\mcC)$ is a diagram, then
    the limit $\mcF = \lim_i \mcF_i$ as presheaves is a sheaf.
    For this we show that $\mcF(\mcU) \cong \mcF(U)$ for any cover $\mcU = \{U_j \to U\}$.
    Take $(s_j)_{j\in J} \in \mcF(\mcU)$, then by definition of the
    limit, we can project these to elements $(s_{ij})_{j \in J} \in \mcF_i(\mcU)$.
    Since each $\mcF_i$ is a sheaf, we have unique elements $s_i \in \mcF_i(U)$
    such that $s_i |_{U_j} = s_{ij}$.

    We now want an element $s\in \mcF(U)$ with projections equal to the $s_i$.
    Choosing an element of $\mcF(U)$, is the same as giving a map $\{*\} \to \mcF(U)$,
    which by the universal property of the limit is the same as giving a cone
    $(\{*\}, \lambda_i)$. Let $\lambda_i(*) = s_i$, then we just need to verify
    that this defines a cone.
    We need that for $f\colon i\to i'$ in $\mcI$,
    $\mcF(f)(s_i) = s_{i'}$. This follows, as $s_i|_{U_j}$ is mapped to $s_{i'}|_{U_j}$
    for all $j\in J$, and hence $s_i$ is mapped to $s_{i'}$ because
    $\mcF_{i'}$ is a sheaf. So we have a unique $s\in \mcF(U)$,
    and by the universal property of the limit
    \begin{equation*}
        s|_{U_j} = s_j \iff s_i|_{U_j} = s_{ij}\; \forall i\in I,
    \end{equation*}
    which holds by construction.
    Hence, the claim holds, and $(-)^\sharp$ is also right exact as a functor
    into sheaves.
\end{proof}
To get a better understanding of sheafification, let us work out
the process of sheafification for some presheaves.
\begin{example}
    We consider the presheaf $\mcF$ of \cref{exmp:bad_presheaves}.
    To calculate $\mcF^+$ we use that
    \begin{equation*}
        \mcF^+(U) = \Bigl(\coprod\limits_{\mcU \in \Cov(U)}\mcF(\mcU)\Bigr)/ \sim.
    \end{equation*}
    If $U$ is non-empty, then
    \begin{equation*}
        \mcF(\mcU) = \{(y_i)_{i\in I}\in \prod_{i\in I} Y \mid y_i = y_j\; \forall i,j \in I\}
        = \{(y)_{i\in I}\mid y\in Y\}
    \end{equation*}
    for any cover $\mcU$ of $U$. Take $(y)_{i\in I} \in \mcF(\mcU)$, and
    $(y')_{j\in J} \in \mcF(\mcU')$. For any common refinement $\mcV$ of
    $\mcU$ and $\mcU'$, we have $(y_{\alpha(i)})|_{V_i} = y$ and $(y'_{\beta(i)})|_{V_i} = y'$
    so that $(y)_{i\in I} \sim (y')_{j\in J} \iff y = y'$.
    So, we find again that $\mcF^+(U) = Y$ as long as $U \neq \emptyset$.

    Now, when $U = \emptyset$ there are two covers: $\{\emptyset \to \emptyset\}$,
    and the empty covering, $\{\}_{i\in \emptyset}$. We have
    $\mcF(\{\emptyset \to \emptyset\}) = \mcF(\emptyset) = Y$,
    while $\mcF(\{\}_{i\in \emptyset}) = \{*\}$.
    Now every $y\in Y$ is equivalent to $*$, as $\{\}_{i\in \emptyset}$
    is a common refinement of the two covers, and the equivalence
    condition becomes an empty statement in this case. As such we find
    $\mcF^+(\emptyset) = \{*\}$.
    Note that $\mcF^+$ is isomorphic to $\mcG$ from \cref{exmp:bad_presheaves}.
    After all, a constant function $U \to Y$ is the same
    as choosing an element in $Y$.

    We now have a separated presheaf $\mcF^+$. What does the sheaf
    $\mcF^\sharp$ look like? Let us first consider the case that $U$
    is empty. Then for both covers $\{\emptyset \to \emptyset\}$ and
    $\{\}_{i\in \emptyset}$ we get $\mcF^+(\mcU) = \{*\}$.
    So $\mcF^\sharp(\emptyset) = \{*\}$.

    More interesting is what happens when $U \neq \emptyset$.
    Let $\{U_i \to U\}_{i\in I}$ be a cover such that
    none of the $U_i$ are empty. Then
    \begin{align*}
        \mcF^+(\mcU) & =
        \{(y_i)_{i\in I}\in \prod_{i\in I} Y \mid
        y_i|_{U_i \cap U_j} = y_j|_{U_i \cap U_j}\; \forall i,j \in I\} \\
                     & = \{(y_i)_{i\in I}\in \prod_{i\in I}Y\mid
        y_i = y_j,\, U_i \cap U_j \neq \emptyset\},
    \end{align*}
    as $\mcF^+(\emptyset) = \{*\}$ and hence $y|_{U_i \cap U_j} = *$
    if $U_i \cap U_j = \emptyset$ for any $y \in Y$.
    We can view elements of $\mcF^+(\mcU)$ as functions $s\colon U \to Y$,
    that are constant on each of the opens $U_i$.
    Adding empty sets
    to the cover does not really change anything: the elements of
    $\mcF(\mcU)$ will be $*$ at the indices for which $U_i = \emptyset$.
    Under $\sim$ such functions will all be the same.
    Take, $f \in \mcF^\sharp(U)$. Then there is some cover
    $\mcU = \{U_i\}_{i\in I}$ such that $f$
    arises from $\mcF^+(\mcU)$. Consequently, for each $x\in U$,
    there is some $U_i$ containing $x$, such that $f$
    is constant on $U_i$. In other words, $f$ is a locally
    constant function on $U$. If we equip $Y$ with the discrete
    topology then this is equivalent to saying that $f\colon U \to Y$
    is continuous.

    The sheaf $\mcF^\sharp$ is (perhaps a little confusingly) called the
    \emph{constant sheaf with value $Y$}.
\end{example}
\section{Sheaf Cohomology}

In the previous section we have explored some
fundamental properties of sheaves of sets.
We will now be looking at sheaves of abelian groups,
which form a very interesting and rich abelian category.
\section{Condensed Sets}
\bibliographystyle{alpha}
\bibliography{references.bib}
\end{document}
