\documentclass{article}

%% PACKAGES

\usepackage{amsmath,amsfonts,amsthm,amssymb,mathtools}
\usepackage[shortlabels]{enumitem}
\usepackage{hyperref}
\usepackage{cleveref}
\usepackage{tikz-cd}

%% Theorem environments
\theoremstyle{plain}
\newtheorem{theorem}{Theorem}[section]
\newtheorem{lemma}[theorem]{Lemma}
\newtheorem{prop}[theorem]{Proposition}
\newtheorem{corollary}[theorem]{Corollary}
\newtheorem*{theorem*}{Theorem}
\newtheorem*{lemma*}{Lemma}
\newtheorem*{prop*}{Proposition}
\newtheorem*{corollary*}{Corollary}

\theoremstyle{definition}
\newtheorem{definition}[theorem]{Definition}
\newtheorem{example}[theorem]{Example}
\newtheorem{exercise}[theorem]{Exercise}
\newtheorem*{definition*}{Definition}
\newtheorem*{example*}{Example}
\newtheorem*{exercise*}{Exercise}

\theoremstyle{remark}
\newtheorem{remark}[theorem]{Remark}
\newtheorem*{remark*}{Remark}

%===========================================================%
%The code below customises theorem numbering
%===========================================================%

\theoremstyle{plain}
\newtheorem{innercustomgenericplain}{\customgenericname}
\providecommand{\customgenericname}{}
\newcommand{\newcustomtheoremplain}[2]{%
    \crefname{#2}{#2}{#2s}%
    \newenvironment{#1}[1]
    {%
        \renewcommand\customgenericname{#2}%
        \crefalias{innercustomgenericplain}{#2}%
        \renewcommand\theinnercustomgenericplain{##1}%
        \innercustomgenericplain
    }
    {\endinnercustomgenericplain}
}

\theoremstyle{definition}
\newtheorem{innercustomgenericdefinition}{\customgenericname}
\providecommand{\customgenericname}{}
\newcommand{\newcustomtheoremdefinition}[2]{%
    \crefname{#2}{#2}{#2s}%
    \newenvironment{#1}[1]
    {%
        \renewcommand\customgenericname{#2}%
        \crefalias{innercustomgenericdefinition}{#2}%
        \renewcommand\theinnercustomgenericdefinition{##1}%
        \innercustomgenericdefinition
    }
    {\endinnercustomgenericdefinition}
}

\theoremstyle{remark}
\newtheorem{innercustomgenericremark}{\customgenericname}
\providecommand{\customgenericname}{}
\newcommand{\newcustomtheoremremark}[2]{%
    \crefname{#2}{#2}{#2s}%
    \newenvironment{#1}[1]
    {%
        \renewcommand\customgenericname{#2}%
        \crefalias{innercustomgenericremark}{#2}%
        \renewcommand\theinnercustomgenericremark{##1}%
        \innercustomgenericremark
    }
    {\endinnercustomgenericremark}
}


\newcustomtheoremplain{customtheorem}{Theorem}
\newcustomtheoremplain{customlemma}{Lemma}
\newcustomtheoremplain{customprop}{Proposition}
\newcustomtheoremplain{customcorollary}{Corollary}

\newcustomtheoremdefinition{customdefinition}{Definition}
\newcustomtheoremdefinition{customexample}{Example}
\newcustomtheoremdefinition{customexercise}{Exercise}

\newcustomtheoremremark{customremark}{Remark}

\newcommand{\inv}{^{-1}}
\newcommand{\defeq}{\overset{\mathrm{def}}{=}}


\newcommand{\liff}{\leftrightarrow}
\newcommand{\lthen}{\rightarrow}
\newcommand{\opname}{\operatorname}
\newcommand{\surjto}{\twoheadrightarrow}
\newcommand{\injto}{\hookrightarrow}
\DeclareMathOperator{\img}{im} % Image
\DeclareMathOperator{\Img}{Im} % Image
\DeclareMathOperator{\coker}{coker} % Cokernel
\DeclareMathOperator{\Coker}{Coker} % Cokernel
\DeclareMathOperator{\Ker}{Ker} % Kernel
\DeclareMathOperator{\rank}{rank} % rank
\DeclareMathOperator{\Spec}{Spec} % spectrum
\DeclareMathOperator{\Tr}{Tr} % trace
\DeclareMathOperator{\pr}{pr} % projection
\DeclareMathOperator{\ext}{ext} % extension
\DeclareMathOperator{\pred}{pred} % predecessor
\DeclareMathOperator{\dom}{dom} % domain
\DeclareMathOperator{\cod}{cod} % codomain
\DeclareMathOperator{\ran}{ran} % range
\DeclareMathOperator{\Hom}{Hom} % homomorphism
\DeclareMathOperator{\Mor}{Mor} % morphisms
\DeclareMathOperator{\ob}{ob} % objects
\DeclareMathOperator{\mor}{mor} % morphisms
\DeclareMathOperator{\Fun}{Fun} % functors
\DeclareMathOperator{\Nat}{Nat} % natural transformations
\DeclareMathOperator{\End}{End} % endomorphism
\DeclareMathOperator{\Ann}{Ann} % annihilator

% Category Theory
\DeclareMathOperator{\Mod}{\mathbf{Mod}}
\DeclareMathOperator{\Top}{\mathbf{Top}}
\DeclareMathOperator{\Vect}{\mathbf{Vect}}
\DeclareMathOperator{\Ring}{\mathbf{Ring}}
\DeclareMathOperator{\Rng}{\mathbf{Rng}}
\DeclareMathOperator{\Ab}{\mathbf{Ab}}
\DeclareMathOperator{\Set}{\mathbf{Set}}
\DeclareMathOperator{\Sh}{\mathbf{Sh}}
\DeclareMathOperator{\PSh}{\mathbf{PSh}}

\newcommand{\ol}{\overline}
\newcommand{\ul}{\underline}
\newcommand{\wt}{\widetilde}
\newcommand{\wh}{\widehat}
\newcommand{\norm}[1]{\left\| #1 \right\|}
\newcommand{\inner}[2]{\left\langle #1 , #2 \right\rangle}


% Things Lie
\newcommand{\kb}{\mathfrak b}
\newcommand{\kg}{\mathfrak g}
\newcommand{\kh}{\mathfrak h}
\newcommand{\kn}{\mathfrak n}
\newcommand{\ku}{\mathfrak u}
\newcommand{\kz}{\mathfrak z}
\DeclareMathOperator{\Ext}{Ext} % Ext functor
\DeclareMathOperator{\Tor}{Tor} % Tor functor
\newcommand{\gl}{\opname{\mathfrak{gl}}} % frak gl group
\renewcommand{\sl}{\opname{\mathfrak{sl}}} % frak sl group chktex 6

% More script letters etc.
\newcommand{\SA}{\mathcal A}
\newcommand{\SB}{\mathcal B}
\newcommand{\SC}{\mathcal C}
\newcommand{\SF}{\mathcal F}
\newcommand{\SG}{\mathcal G}
\newcommand{\SH}{\mathcal H}
\newcommand{\OO}{\mathcal O}

\newcommand{\SCA}{\mathscr A}
\newcommand{\SCB}{\mathscr B}
\newcommand{\SCC}{\mathscr C}
\newcommand{\SCD}{\mathscr D}
\newcommand{\SCE}{\mathscr E}
\newcommand{\SCF}{\mathscr F}
\newcommand{\SCG}{\mathscr G}
\newcommand{\SCH}{\mathscr H}

% Mathfrak primes
\newcommand{\km}{\mathfrak m}
\newcommand{\kp}{\mathfrak p}
\newcommand{\kq}{\mathfrak q}

% number sets
\newcommand{\RR}[1][]{\ensuremath{\ifstrempty{#1}{\mathbb{R}}{\mathbb{R}^{#1}}}}
\newcommand{\NN}[1][]{\ensuremath{\ifstrempty{#1}{\mathbb{N}}{\mathbb{N}^{#1}}}}
\newcommand{\ZZ}[1][]{\ensuremath{\ifstrempty{#1}{\mathbb{Z}}{\mathbb{Z}^{#1}}}}
\newcommand{\QQ}[1][]{\ensuremath{\ifstrempty{#1}{\mathbb{Q}}{\mathbb{Q}^{#1}}}}
\newcommand{\CC}[1][]{\ensuremath{\ifstrempty{#1}{\mathbb{C}}{\mathbb{C}^{#1}}}}
\newcommand{\PP}[1][]{\ensuremath{\ifstrempty{#1}{\mathbb{P}}{\mathbb{P}^{#1}}}}
\newcommand{\HH}[1][]{\ensuremath{\ifstrempty{#1}{\mathbb{H}}{\mathbb{H}^{#1}}}}
\newcommand{\FF}[1][]{\ensuremath{\ifstrempty{#1}{\mathbb{F}}{\mathbb{F}^{#1}}}}
% expected value
\newcommand{\EE}{\ensuremath{\mathbb{E}}}
\newcommand{\charin}{\text{ char }}
\DeclareMathOperator{\sign}{sign}
\DeclareMathOperator{\Aut}{Aut}
\DeclareMathOperator{\Inn}{Inn}
\DeclareMathOperator{\Syl}{Syl}
\DeclareMathOperator{\Gal}{Gal}
\DeclareMathOperator{\GL}{GL} % General linear group
\DeclareMathOperator{\SL}{SL} % Special linear group

%---------------------------------------
% BlackBoard Math Fonts :-
%---------------------------------------

%Captital Letters
\newcommand{\bbA}{\mathbb{A}}	\newcommand{\bbB}{\mathbb{B}}
\newcommand{\bbC}{\mathbb{C}}	\newcommand{\bbD}{\mathbb{D}}
\newcommand{\bbE}{\mathbb{E}}	\newcommand{\bbF}{\mathbb{F}}
\newcommand{\bbG}{\mathbb{G}}	\newcommand{\bbH}{\mathbb{H}}
\newcommand{\bbI}{\mathbb{I}}	\newcommand{\bbJ}{\mathbb{J}}
\newcommand{\bbK}{\mathbb{K}}	\newcommand{\bbL}{\mathbb{L}}
\newcommand{\bbM}{\mathbb{M}}	\newcommand{\bbN}{\mathbb{N}}
\newcommand{\bbO}{\mathbb{O}}	\newcommand{\bbP}{\mathbb{P}}
\newcommand{\bbQ}{\mathbb{Q}}	\newcommand{\bbR}{\mathbb{R}}
\newcommand{\bbS}{\mathbb{S}}	\newcommand{\bbT}{\mathbb{T}}
\newcommand{\bbU}{\mathbb{U}}	\newcommand{\bbV}{\mathbb{V}}
\newcommand{\bbW}{\mathbb{W}}	\newcommand{\bbX}{\mathbb{X}}
\newcommand{\bbY}{\mathbb{Y}}	\newcommand{\bbZ}{\mathbb{Z}}

%---------------------------------------
% MathCal Fonts :-
%---------------------------------------

%Captital Letters
\newcommand{\mcA}{\mathcal{A}}	\newcommand{\mcB}{\mathcal{B}}
\newcommand{\mcC}{\mathcal{C}}	\newcommand{\mcD}{\mathcal{D}}
\newcommand{\mcE}{\mathcal{E}}	\newcommand{\mcF}{\mathcal{F}}
\newcommand{\mcG}{\mathcal{G}}	\newcommand{\mcH}{\mathcal{H}}
\newcommand{\mcI}{\mathcal{I}}	\newcommand{\mcJ}{\mathcal{J}}
\newcommand{\mcK}{\mathcal{K}}	\newcommand{\mcL}{\mathcal{L}}
\newcommand{\mcM}{\mathcal{M}}	\newcommand{\mcN}{\mathcal{N}}
\newcommand{\mcO}{\mathcal{O}}	\newcommand{\mcP}{\mathcal{P}}
\newcommand{\mcQ}{\mathcal{Q}}	\newcommand{\mcR}{\mathcal{R}}
\newcommand{\mcS}{\mathcal{S}}	\newcommand{\mcT}{\mathcal{T}}
\newcommand{\mcU}{\mathcal{U}}	\newcommand{\mcV}{\mathcal{V}}
\newcommand{\mcW}{\mathcal{W}}	\newcommand{\mcX}{\mathcal{X}}
\newcommand{\mcY}{\mathcal{Y}}	\newcommand{\mcZ}{\mathcal{Z}}


%---------------------------------------
% Bold Math Fonts :-
%---------------------------------------

%Captital Letters
\newcommand{\bmA}{\boldsymbol{A}}	\newcommand{\bmB}{\boldsymbol{B}}
\newcommand{\bmC}{\boldsymbol{C}}	\newcommand{\bmD}{\boldsymbol{D}}
\newcommand{\bmE}{\boldsymbol{E}}	\newcommand{\bmF}{\boldsymbol{F}}
\newcommand{\bmG}{\boldsymbol{G}}	\newcommand{\bmH}{\boldsymbol{H}}
\newcommand{\bmI}{\boldsymbol{I}}	\newcommand{\bmJ}{\boldsymbol{J}}
\newcommand{\bmK}{\boldsymbol{K}}	\newcommand{\bmL}{\boldsymbol{L}}
\newcommand{\bmM}{\boldsymbol{M}}	\newcommand{\bmN}{\boldsymbol{N}}
\newcommand{\bmO}{\boldsymbol{O}}	\newcommand{\bmP}{\boldsymbol{P}}
\newcommand{\bmQ}{\boldsymbol{Q}}	\newcommand{\bmR}{\boldsymbol{R}}
\newcommand{\bmS}{\boldsymbol{S}}	\newcommand{\bmT}{\boldsymbol{T}}
\newcommand{\bmU}{\boldsymbol{U}}	\newcommand{\bmV}{\boldsymbol{V}}
\newcommand{\bmW}{\boldsymbol{W}}	\newcommand{\bmX}{\boldsymbol{X}}
\newcommand{\bmY}{\boldsymbol{Y}}	\newcommand{\bmZ}{\boldsymbol{Z}}
%Small Letters
\newcommand{\bma}{\boldsymbol{a}}	\newcommand{\bmb}{\boldsymbol{b}}
\newcommand{\bmc}{\boldsymbol{c}}	\newcommand{\bmd}{\boldsymbol{d}}
\newcommand{\bme}{\boldsymbol{e}}	\newcommand{\bmf}{\boldsymbol{f}}
\newcommand{\bmg}{\boldsymbol{g}}	\newcommand{\bmh}{\boldsymbol{h}}
\newcommand{\bmi}{\boldsymbol{i}}	\newcommand{\bmj}{\boldsymbol{j}}
\newcommand{\bmk}{\boldsymbol{k}}	\newcommand{\bml}{\boldsymbol{l}}
\newcommand{\bmm}{\boldsymbol{m}}	\newcommand{\bmn}{\boldsymbol{n}}
\newcommand{\bmo}{\boldsymbol{o}}	\newcommand{\bmp}{\boldsymbol{p}}
\newcommand{\bmq}{\boldsymbol{q}}	\newcommand{\bmr}{\boldsymbol{r}}
\newcommand{\bms}{\boldsymbol{s}}	\newcommand{\bmt}{\boldsymbol{t}}
\newcommand{\bmu}{\boldsymbol{u}}	\newcommand{\bmv}{\boldsymbol{v}}
\newcommand{\bmw}{\boldsymbol{w}}	\newcommand{\bmx}{\boldsymbol{x}}
\newcommand{\bmy}{\boldsymbol{y}}	\newcommand{\bmz}{\boldsymbol{z}}

%---------------------------------------
% Scr Math Fonts :-
%---------------------------------------

\newcommand{\sA}{{\mathscr{A}}}   \newcommand{\sB}{{\mathscr{B}}}
\newcommand{\sC}{{\mathscr{C}}}   \newcommand{\sD}{{\mathscr{D}}}
\newcommand{\sE}{{\mathscr{E}}}   \newcommand{\sF}{{\mathscr{F}}}
\newcommand{\sG}{{\mathscr{G}}}   \newcommand{\sH}{{\mathscr{H}}}
\newcommand{\sI}{{\mathscr{I}}}   \newcommand{\sJ}{{\mathscr{J}}}
\newcommand{\sK}{{\mathscr{K}}}   \newcommand{\sL}{{\mathscr{L}}}
\newcommand{\sM}{{\mathscr{M}}}   \newcommand{\sN}{{\mathscr{N}}}
\newcommand{\sO}{{\mathscr{O}}}   \newcommand{\sP}{{\mathscr{P}}}
\newcommand{\sQ}{{\mathscr{Q}}}   \newcommand{\sR}{{\mathscr{R}}}
\newcommand{\sS}{{\mathscr{S}}}   \newcommand{\sT}{{\mathscr{T}}}
\newcommand{\sU}{{\mathscr{U}}}   \newcommand{\sV}{{\mathscr{V}}}
\newcommand{\sW}{{\mathscr{W}}}   \newcommand{\sX}{{\mathscr{X}}}
\newcommand{\sY}{{\mathscr{Y}}}   \newcommand{\sZ}{{\mathscr{Z}}}


%---------------------------------------
% Math Fraktur Font
%---------------------------------------

%Captital Letters
\newcommand{\mfA}{\mathfrak{A}}	\newcommand{\mfB}{\mathfrak{B}}
\newcommand{\mfC}{\mathfrak{C}}	\newcommand{\mfD}{\mathfrak{D}}
\newcommand{\mfE}{\mathfrak{E}}	\newcommand{\mfF}{\mathfrak{F}}
\newcommand{\mfG}{\mathfrak{G}}	\newcommand{\mfH}{\mathfrak{H}}
\newcommand{\mfI}{\mathfrak{I}}	\newcommand{\mfJ}{\mathfrak{J}}
\newcommand{\mfK}{\mathfrak{K}}	\newcommand{\mfL}{\mathfrak{L}}
\newcommand{\mfM}{\mathfrak{M}}	\newcommand{\mfN}{\mathfrak{N}}
\newcommand{\mfO}{\mathfrak{O}}	\newcommand{\mfP}{\mathfrak{P}}
\newcommand{\mfQ}{\mathfrak{Q}}	\newcommand{\mfR}{\mathfrak{R}}
\newcommand{\mfS}{\mathfrak{S}}	\newcommand{\mfT}{\mathfrak{T}}
\newcommand{\mfU}{\mathfrak{U}}	\newcommand{\mfV}{\mathfrak{V}}
\newcommand{\mfW}{\mathfrak{W}}	\newcommand{\mfX}{\mathfrak{X}}
\newcommand{\mfY}{\mathfrak{Y}}	\newcommand{\mfZ}{\mathfrak{Z}}
%Small Letters
\newcommand{\mfa}{\mathfrak{a}}	\newcommand{\mfb}{\mathfrak{b}}
\newcommand{\mfc}{\mathfrak{c}}	\newcommand{\mfd}{\mathfrak{d}}
\newcommand{\mfe}{\mathfrak{e}}	\newcommand{\mff}{\mathfrak{f}}
\newcommand{\mfg}{\mathfrak{g}}	\newcommand{\mfh}{\mathfrak{h}}
\newcommand{\mfi}{\mathfrak{i}}	\newcommand{\mfj}{\mathfrak{j}}
\newcommand{\mfk}{\mathfrak{k}}	\newcommand{\mfl}{\mathfrak{l}}
\newcommand{\mfm}{\mathfrak{m}}	\newcommand{\mfn}{\mathfrak{n}}
\newcommand{\mfo}{\mathfrak{o}}	\newcommand{\mfp}{\mathfrak{p}}
\newcommand{\mfq}{\mathfrak{q}}	\newcommand{\mfr}{\mathfrak{r}}
\newcommand{\mfs}{\mathfrak{s}}	\newcommand{\mft}{\mathfrak{t}}
\newcommand{\mfu}{\mathfrak{u}}	\newcommand{\mfv}{\mathfrak{v}}
\newcommand{\mfw}{\mathfrak{w}}	\newcommand{\mfx}{\mathfrak{x}}
\newcommand{\mfy}{\mathfrak{y}}	\newcommand{\mfz}{\mathfrak{z}}


\usepackage[width=400pt]{geometry}

\DeclareMathOperator{\Open}{\mathbf{Open}}
\DeclareMathOperator{\opp}{opp}
\DeclareMathOperator{\Cont}{Cont}
\DeclareMathOperator{\Cov}{Cov}
\DeclareMathOperator{\colim}{colim}
\DeclareMathOperator{\id}{id}
\DeclareMathOperator{\Cond}{Cond}

\title{A Fluid Introduction to Condensed Mathematics}
\author{Wannes Malfait}
\date{}

\begin{document}

\maketitle
\newpage
\tableofcontents

\section{Introduction}
Most of the contents of this paper will be based on the lecture
notes by Peter Scholze, who developed the theory of condensed mathematics
together with Dustin Clausen (\cite{Sch2019LecturesCM}).
We first develop some necessary background from algebraic geometry
relating to sheaves. This theory is then used to define sheaf
cohomology on topological spaces. Finally, we give a short
introduction to condensed mathematics itself, giving the basic
definitions and exploring some examples. In the end we look
at cohomology in the condensed setting, and compare it to sheaf
cohomology. Although there are no original results, a lot of proofs
from the references have been written out in more detail, and
worked out examples illustrating different concepts have been
added throughout the whole paper.

\section{Sheaves}
A central role in this paper will be played by sheaves.
It therefore makes sense to study these objects more closely,
and attempt to gain some intuition before jumping straight into
the heart of the matter.
The first step will be to convince ourselves that sheaves
are interesting enough to study, and that they form a nice type
of category. As such, we will look at sheaf cohomology,
and compare it to singular cohomology. But first, we must
define what a sheaf is.

\subsection{On topological spaces}
Consider the following situation in algebraic geometry.
We have some (affine) algebraic variety $X$, with the Zariski Topology.
Associated to $X$ is an ideal $I$ of all the polynomials that vanish
on $X$. Then, the quotient ring $\mcO_X(X) = k[x_1,\ldots,x_n]/I$, is
the coordinate ring of $X$. The elements of $\mcO_X(X)$ are the regular
functions on $X$. For an open $U\subseteq X$ we can also consider the
regular functions on $U$, which we will denote as $\mcO_X(U)$.
This mapping between open subsets of $X$ and rings,
given by $\mcO_X$, is precisely a sheaf.
In this case, $\mcO_X$ is known as the \emph{structure sheaf}.
In general, a sheaf will give some global data that can be defined
locally.
\begin{definition}[Presheaves]
    Let $X$ be a topological space.
    A \emph{presheaf of sets} $\mcF$ on $X$ consists of two things:
    \begin{enumerate}
        \item For each open $U\subseteq X$ a set $\mcF(U)$.
              These are known as the \emph{sections} of $\mcF$ over $U$.
        \item For each inclusion $U\subseteq V$ a map
              $\rho_{U,V}\colon \mcF(V) \to \mcF(U)$, such that $\rho_{U,U} = id_{\mcF(U)}$, and
              for $U\subseteq V \subseteq W$ we have $\rho_{U,V}\circ \rho_{V,W} = \rho_{U, W}$.
              We call the $\rho_{U,V}$ the \emph{restrictions}, and if $s\in \mcF(U)$,
              we will often denote $\rho_{V,U}(s) = s|_V$.
    \end{enumerate}
\end{definition}
\begin{remark}
    If we write $\Open(X)$ for the category of open sets of $X$,
    with morphisms given by the inclusion maps, then a presheaf is
    precisely a functor $\mcF\colon \Open(X)^{\opp} \to \Set$.
\end{remark}

This only defines what a ``presheaf'' is.
The sheaf condition will ensure that the value of $\mcF(U)$ can be
constructed by defining it locally on elements of a cover $\{U_i\}_{i\in I}$ of $U$.
It is exactly this interaction between local definitions and global objects
that makes sheaves so useful.
\begin{definition}[Sheaves]
    Let $X$ be a topological space, and $\mcF$ a presheaf of sets on $X$.
    We say that $\mcF$ is a sheaf, if it satisfies the following additional properties
    for any open $U\subseteq X$ and open cover $\{U_i\}_{i\in I}$ of $U$:
    \begin{itemize}
        \item (uniqueness/locality) If $s,t\in \mcF(U)$ are sections such that
              $s|_{U_i} = t|_{U_i}$ for all $i\in I$, then $s=t$.
        \item (gluing) If $\{s_i \in \mcF(U_i)\}_{i\in I}$ is
              a collection of sections such that
              \begin{equation*}
                  s_i|_{U_i \cap U_j} = s_j|_{U_i \cap U_j} \text { for all } i, j,
              \end{equation*}
              then there is a section $s\in \mcF(U)$ such that $s_i = s|_{U_i}$ for all $i \in I$.
    \end{itemize}
    In other words, if we have a bunch of sections
    that agree on overlaps, then we can uniquely glue them together.
\end{definition}
\begin{remark}
    \label{rem:cat_def_sheaf}
    We can reformulate the properties in a more categorical manner.
    Namely, for any open cover $\{U_i\}_{i\in I}$ of $U\subseteq X$,
    the following diagram:
    \begin{equation*}
        \begin{tikzcd}[cramped]
            \mcF(U) \rar
            &\prod\limits_{i\in I} \mcF(U_i) \rar[shift left] \rar[shift right]
            & \prod\limits_{i,j \in I} \mcF(U_i\cap U_j),
        \end{tikzcd}
    \end{equation*}
    has to be an equalizer. This diagram makes sense in any target category
    that has all limits. In terms of sets,
    the first map is given by $s \mapsto (s|_{U_i})_{i\in I}$. The pair of
    maps is given by the two possible restrictions, $(s_i)_{i\in I} \mapsto ((s_i|_{U_i \cap U_j})_{j\in I})_{i\in I}$
    and $(s_j)_{j\in I} \mapsto ((s_j|_{U_i \cap U_j})_{i\in I})_{j\in J}$.

    In this way, we can define sheaves on rings or abelian groups, or other
    categories with all (set-indexed) limits.
    For example, a presheaf of abelian groups on $X$, is a functor
    $\mcF \colon \Open(X)^{\opp} \to \Ab$.
\end{remark}

In many cases, defining some structure on sheaves, will come down
to defining it on $\mcF(U)$ in a compatible way.
As an example, we can look at morphisms of sheaves.
\begin{definition}[Morphisms of sheaves]
    Let $\mcF, \mcG$ be presheaves on a topological space $X$.
    A  \emph{morphism of presheaves} $\varphi \colon \mcF \to \mcG$ assigns to
    each open $U\subset X$ a morphism $\varphi_U \colon \mcF(U) \to \mcG(U)$
    compatible with the restriction maps, i.e. for $U\subseteq V \subseteq X$
    the following diagram commutes:
    \begin{equation*}
        \begin{tikzcd}
            \mcF(V) \rar["\varphi_V"] \dar["\rho^\mcF_{U,V}"]
            & \mcG(V) \dar["\rho^\mcG_{U,V}"] \\
            \mcF(U) \rar["\varphi_U"]
            & \mcG(U)
        \end{tikzcd}.
    \end{equation*}
    In other words, $\varphi$ is a natural transformation between $\mcF$ and $\mcG$.
    Using a bit of abusive notation, the compatibility can be read as
    $\varphi(s|_U) = \varphi(s)|_U$.
    A \emph{morphism of sheaves} is a morphism of the underlying presheaves.
\end{definition}

For a topological space $X$, we now have (pre)sheaves and morphisms between them.
These form a category. We will write $\PSh(X)$ for the category of presheaves of
sets on $X$, and $\Sh(X)$ for the category of sheaves of sets on $X$.
In the same way, we write $\Ab(X), \Ring(X), \Vect(X),\dots $ for the category of
sheaves of abelian groups, rings, vector spaces, \dots on $X$.
\begin{remark}
    We can "recover" the underlying space, by taking $X = \{*\}$,
    the one-point space. We have $\Sh(*) = \Set, \Ab(*) = \Ab, \dots$
\end{remark}

\subsubsection{Examples}
To solidify our understanding of sheaves, it will be beneficial to look
at some examples.
\begin{example}
    The simplest examples of sheaves are those were
    $\mcF(U)$ is not just a set, but a set of functions, and the
    restrictions correspond to actual restrictions.

\end{example}
\begin{example}
    Let $Y$ be another topological space, then we can define $\mcF(U) \defeq \Cont(U, Y)$
    where $\Cont(U,Y) = \{f\colon U \to Y \mid f \text{ continuous }\}$.
    The restrictions are defined as actual restrictions, and we get a presheaf. To see that
    it is a sheaf, we need to verify the gluing condition.
    Let $\{U_i\}$ be an open covering of $U\subseteq X$,
    and $f_i \colon U_i \to Y$ is a continuous map for each $i\in I$,
    such that $f_i|_{U_i \cap U_j} = f_j |_{U_i \cap U_j}$ for all $i,j\in I$.
    We can now define a map $f\colon U \to Y$ via $f(u) = f_i(u)$ for any $i\in I$
    such that $u\in U_i$. By assumption, this map is well-defined.
    To see that it is continuous, for any $V\subseteq Y$ open,
    we can write
    \begin{equation*}
        f\inv(V) = U \cap f\inv(V)=
        \bigcup_{i\in I} (U_i \cap f\inv(V)) =
        \bigcup_{i\in I}f_i\inv(V),
    \end{equation*}
    which is open since all the $f_i$ are continuous.

    So we see that $\mcF$ is a sheaf.
    In the case that $Y$ has the discrete topology,
    we call $\mcF$ the \emph{constant sheaf with value Y}.
    The sections of $\mcF$ over $U$ are the \emph{locally} constant functions,
    i.e. at each point $x\in U$ we can find an open neighborhood $V$ of $x$,
    such that $f$ is constant on $V$.
\end{example}
\begin{example}

    Let $f\colon X\to Y$ be a continuous map, then we get a sheaf
    on $X$ by the rule
    \begin{equation*}
        \Gamma(Y/X)(U) = \{s\colon U\to Y \mid f\circ s = 1_{U}\}.
    \end{equation*}
    The gluing construction is the same as in the previous example.
    To see that this yields another section, note that for
    any $u \in U$, $s(u) = s_i(u)$ for some $i\in I$, and hence
    $f(s(u)) = f(s_i(u)) = u$.
    We call $\Gamma(Y/X)$ the \emph{sheaf of sections of $f$}.
\end{example}
\begin{example}
    As we remarked above, the structure sheaf $\mcO_X$ is also a sheaf,
    where again, the restrictions are actual restrictions of functions.
    To show that the gluing of regular functions is again regular,
    requires some machinery from algebraic geometry, which falls outside
    the scope of this paper.
\end{example}

As it turns out, it is possible to generalize
the definition of sheaves to categories which
also have the notion of a covering. We will
need this notion when discussing condensed sets.
\subsection{On a site}

The idea of a site was first introduced by
Alexander Grothendieck, and has proven to be
very useful in algebraic geometry.
For this subsection we largely follow
\cite[\href{https://stacks.math.columbia.edu/tag/00UZ}{Tag 00UZ}]{stacks-project}.

\subsubsection{Coverings and sites}

To define a site, we collect all the essential
properties that coverings of topological spaces have:
\begin{definition}[Coverings and Sites]
    A \emph{site} is a small category $\mcC$ together
    with a set $\Cov(\mcC)$ of \emph{coverings} of
    $\mcC$, where the following axioms hold:
    \footnote{We force $\Cov(\mcC)$ to be a set since we will take limits
        over all coverings. It is possible to let $\Cov(\mcC)$ be a class,
        and then show that it can be replaced with a set of coverings
        that gives rise to the same category of sheaves.
        See \cite[\href{https://stacks.math.columbia.edu/tag/00VI}{Tag 00VI}]{stacks-project}.}
    \begin{enumerate}
        \item If $V \to U$ is an isomorphism, then $\{V \to U\}\in \Cov(\mcC)$.
        \item If $\{U_i \to U\}_{i\in I}$ is a covering, and for each $i\in I$,
              $\{V_{ij} \to U_i\}_{j \in J_i}$ is a covering as well, then so is the
              composition $\{V_{ij} \to U\}_{i\in I, j\in J_i}$.
        \item If $\{U_i \to U\}_{i\in I}$ is a covering and $V \to U$ is a morphism
              of $\mcC$, then the pullback $U_i\times_U V$ exists for all $i$
              and $\{U_i\times_U V \to V\}_{i\in I}$ is a covering.
    \end{enumerate}
\end{definition}

To make sense of this definition, let us look at the
canonical example.
\begin{example}
    Let $X$ be a topological space, then $\Open(X)$ is a
    site with coverings given by the open covers.
    Let us verify that the axioms hold:
    \begin{enumerate}
        \item The only isomorphisms in $\Open(X)$ are the identity maps.
              Since $\{U\}$ is an open cover of $U$ for any $U \in \Open(X)$,
              the first axiom is satisfied.
        \item If $U = \bigcup_{i\in I} U_i$ is an open cover of $U$,
              and for each $i \in I$, $U_i = \bigcup_{j\in J_i} U_{ij}$ is an
              open cover of $U_i$, then $U = \bigcup_{i\in I}\bigcup_{j \in J_i} U_{ij}$
              is an open cover of $U$.
        \item If $U = \bigcup_{i\in I} U_i$ is an open cover, and $V \subseteq U$, then
              $V = \bigcup_{i \in I} V \cap U_i$ is an open cover of $V$.
    \end{enumerate}
    Here we used that $U\times_W V = U\cap V$ for $U, V \subseteq W$,
    which follows from $S \subseteq V, S \subseteq U \implies S \subseteq V\cap U$.
\end{example}

The following example is still quite similar to the previous
example. The underlying site of a condensed set will be similar
to this site.
\begin{example}
    Let $G$ be a group, then we can consider the category
    of all $G$-sets, i.e. sets with a corresponding action
    by the group $G$. Morphisms are given by maps $f \colon X \to Y$
    satisfying $g\cdot f(x) = f(g\cdot x)$, i.e. $G$-equivariant
    maps. To make this into a site\footnote{
        Technically, this is not a site, since the category
        of all $G$-sets is a proper class. The example is
        simply meant to illustrate what covers could look like.
        There is a way around this problem though. See
        \cite[\href{https://stacks.math.columbia.edu/tag/00VK}{Example 00VK}]{stacks-project}.
    }, we define the covers to be the
    families of jointly surjective maps. In other words
    $\{f_i \colon X_i \to X\}_{i\in I}$ is a cover if
    $\bigcup_{i\in I}f_i(X_i) = X$. Let us verify the axioms:
    \begin{enumerate}
        \item If $f \colon X \to Y$ is an isomorphism, then
              in particular it is surjective, so $\{f \colon X \to Y\}$ is a cover
              of $Y$
        \item If $\{f_i \colon X_i \to Y\}_{i\in I}$ is jointly surjective, and
              for each $i\in I$ the family $\{f_{ij} \colon X_{ij} \to X_i\}_{j\in J_i}$ is jointly
              surjective, then so is $\{f_i \circ f_{ij}\colon X_{ij} \to Y\}_{i\in I, j\in J_i}$.
        \item Let $\{f_i \colon X_i \to Y\}_{i\in I}$ be a cover,
              and $f\colon T \to Y$ a $G$-equivariant map. We claim that the pullback
              $X_i \times_Y T$ exists and is given by
              $S = \{(x,t)\in X_i \times T \mid f_i(x) = f(t)\}$.
              This is a $G$-set, as if $f_i(x) = f(t)$, then
              \begin{equation*}
                  f_i(g\cdot x) = g\cdot f_i(x) = g\cdot f(t) = f(g \cdot t).
              \end{equation*}
              Since $S$ is the pullback in $\Set$ and any $G$-equivariant map
              is in particular a map of sets, the claim follows.
              To see that the projection maps $\{X_i \times_Y T \to T\}$
              are jointly surjective, take $t\in T$. There exists $i\in I$
              such that $f(t) \in f_i(X_i)$, so $(x_i, t) \in X_i \times_Y T$
              for some $x_i \in X_i$.
    \end{enumerate}
\end{example}

In this new context, we have to reword the definitions
of presheaves and sheaves in a more abstract manner.
\begin{definition}[Presheaves and Sheaves on a Site]
    Let $\mcC$ be a small category. A \emph{presheaf on $\mcC$} is
    a contravariant functor $\mcF \colon \mcC \to \Set$.
    If $\mcC$ is a site, then a \emph{sheaf on $\mcC$} is a
    presheaf $\mcF$, such that for any covering $\{U_i \to U\}_{i\in I}$,
    the following diagram is an equalizer:
    \begin{equation*}
        \begin{tikzcd}
            \mcF(U) \rar["e"] &\prod\limits_{i\in I}\mcF(U_i)
            \rar[shift left, "p_0^*"] \rar[shift right, "p_1^*"']
            & \prod\limits_{i,j\in I}\mcF(U_i \times_U U_j)
        \end{tikzcd}.
    \end{equation*}
    The map $e$ is given by $s \mapsto (s|_{U_i})_{i\in I}$.
    For the second maps, take $i,j\in I$. Then we have projections
    $p^{(i,j)}_0 \colon U_i \times_U U_j \to U_i$ and
    $p^{(i,j)}_1 \colon U_i \times_U U_j \to U_j$.
    These result in maps
    $p^{(i,j), *}_0 \colon \mcF(U_i) \to \mcF(U_i \times_U U_j )$ and
    $p^{(i,j), *}_1 \colon \mcF(U_j) \to \mcF(U_i \times_U U_j )$.
    The maps $p_0^*$ and $p_1^*$ are
    then given at component $i,j$ by mapping $(s_k)_{k\in I}$ to
    $p^{(i,j), *}_0(s_i)$ and $p^{(i,j), *}_1(s_j)$ respectively.
\end{definition}

\subsubsection{Sheafification}
In a lot of constructions, the natural thing we do, will end up
being a presheaf, but not a sheaf in general. So, we will need some way to turn
presheaves into sheaves.
\begin{prop}
    \label{prop:sheafification}
    The fully faithful inclusion
    \begin{equation*}
        \iota \colon \Sh(\mcC) \injto \PSh(\mcC),
    \end{equation*}
    admits a left adjoint $\mcF \to \mcF^\sharp$, ``sheafification''.
\end{prop}

We will give an explicit construction of $\mcF^\sharp$.
As it turns out, ``sheafification'' is actually a two-step process.
The first step is making the presheaf separated, and then turning
the separated presheaf into a sheaf.
\begin{definition}
    We say that a presheaf $\mcF$ is \emph{separated} if
    $\mcF(U) \to \prod\limits_{i\in I} \mcF(U_i)$ is injective
    for any cover $\{U_i \to U\}_{i\in I} \in \Cov(\mcC)$.
\end{definition}
\begin{remark}
    Since any equalizer is a monomorphism, it follows from the definition
    of sheaves on a site, that a sheaf is separated.
\end{remark}
\begin{example}
    \label{exmp:bad_presheaves}
    The presheaf $\mcF$ on a topological space $X$, which maps every
    open to the same set $Y \neq \{*\}$ is not separated, as
    \begin{equation*}
        Y = \mcF(\emptyset) \to \prod_{i \in \emptyset} = \{*\},
    \end{equation*}
    is not injective if $Y \neq \{*\}$.

    On the other hand, the presheaf $\mcG$ on a topological space
    $X$ which maps every open $U$ to the constant functions $U \to Y$
    for some space $Y$, is separated but not a sheaf. After all,
    if $U = \cup_i U_i$ is an open cover of $U$, and $f,g \in \mcG(U)$
    are such that $f|_{U_i} = g|_{U_i}$ then $f = g$, since any $x\in U$
    is contained in some $U_i$, and hence $f(x) = f|_{U_i}(x) = g|_{U_i}(x) = g(x)$.
    Since there is only a single function $\emptyset \to Y$ there are no
    problems with empty coverings. $\mcG$ fails to be a sheaf
    in general, since if $U = U_1 \cup U_2$ with $U_1 \cap U_2 = \emptyset$
    and $U_1 \neq \emptyset \neq U_2$, then we can't ``glue'' the functions
    $f \colon U_1 \to Y, f(x) = y_1$ and $g\colon U_2 \to Y, f(x) =y_2$
    together to a constant function on $U$ if $y_1 \neq y_2$.
\end{example}
We first have a few lemmas regarding limits of presheaves and sheaves.
\begin{lemma}
    \label{lem:lims_colims_presheaves}
    Let $\mcC$ be a site, then limits and
    colimits exist in $\PSh(\mcC)$. Additionally, for any $U\in \mcC$,
    the functor $\PSh(\mcC) \to \Set \colon \mcF \mapsto \mcF(U)$
    commutes with all limits and colimits.
\end{lemma}
\begin{proof}
    We only prove the case of limits. The colimit case is
    analogous.

    Let $\mcF \colon \mcI \to \PSh(\mcC)$ be a diagram.
    We get a cone $(\mcF_{\lim}, p^i)$ by
    moving to the images in $\Set$. In other words we take
    $\mcF_{\lim} (U) = \lim_{i \in \mcI} \mcF_i(U)$,
    and $p^i_U \colon \lim_{i\in \mcI}\mcF_i(U) \to \mcF_i(U)$.
    If $U\to V$ is a morphism in $\mcC$, we get a unique map
    $\mcF_{\lim}(V) \to \mcF_{\lim}(U)$, given by the fact
    that $\mcF_{\lim}(U)$ is a limit, and $\mcF_{\lim}(V)$ is a cone
    on all the $\mcF_i(U)$ by composing the maps $\mcF_{\lim}(V) \to \mcF_i(V)$ with
    the maps $\mcF_i(V) \to \mcF_i(U)$. So we see that $\mcF_{\lim}$
    is indeed a presheaf.
    \begin{equation*}
        \begin{tikzcd}[column sep=small]
            &&\mcF_{\lim}(V) \dar[dashed] \ar[ddll, bend right, "p^i_V"'] \ar[ddrr, bend left, "p^{i'}_V"]\\
            &&\mcF_{\lim}(U) \ar[dl, "p^i_U"'] \ar[dr, "p^{i'}_U"]\\
            \mcF_i(V) \rar &\mcF_i(U) \ar[rr] &&\mcF_{i'}(U) &\mcF_{i'}(V) \lar
        \end{tikzcd}
    \end{equation*}
    To see that the maps $p^i$ are morphisms of presheaves, we need
    to verify that the following diagram commutes:
    \begin{equation*}
        \begin{tikzcd}
            \mcF_{\lim}(V) \dar \rar["p^i_V"] & \mcF_i(V) \dar \\
            \mcF_{\lim}(U) \rar["p^i_U"] & \mcF_i(U)
        \end{tikzcd},
    \end{equation*}
    but that already follows from the previous diagram.

    Let us now verify that $\mcF_{\lim}$ is actually a limit.
    If $(\mcG, g^i)$ is another cone, then for each $U\in \mcC$,
    we get a unique map $\mcG(U) \to \mcF_{\lim}(U)$, such that the
    corresponding cone diagrams commute. It suffices to show that these
    maps combine to form a map of presheaves.
    Using the universal property of limits, this comes down to showing
    \begin{equation*}
        \begin{tikzcd}
            \mcG(V)\rar \dar & \mcF_{\lim}(V) \dar \\
            \mcG(U)\rar & \mcF_{\lim}(U)
        \end{tikzcd}
        \iff
        \begin{tikzcd}
            \mcG(V)\rar["g^i_V"] \dar & \mcF_{i}(V) \dar \\
            \mcG(U)\rar["g^i_U"] & \mcF_{i}(U)
        \end{tikzcd}
        \forall i\in \mcI.
    \end{equation*}
    The commutativity of these last diagrams, is just saying that the
    maps $g^i$ are presheaf morphisms, which is true by $(\mcG, g^i)$ being a cone.
\end{proof}
The big difference between sheaves and presheaves, is that we can glue
things together defined on a cover. The trick will be to force
the presheaf to behave as we want. Let us try and make this more precise.

So far we have just talked about coverings as objects. We can
also consider maps between coverings. If $\mcU = \{U_i \to U\}_{i\in I}$
and $\mcV = \{V_j \to V\}_{j \in J}$ are two coverings, a morphism
of coverings between $\mcU$ and $\mcV$ is
a morphism $U \to V$ and a map $\alpha \colon I \to J$
together with morphisms $U_i \to V_{\alpha(i)}$
such that
\begin{equation*}
    \begin{tikzcd}
        U_i \ar[d] \rar & V_{\alpha(i)} \dar\\
        U \rar & V
    \end{tikzcd}
\end{equation*}
commutes. If $U = V$ and $U \to V$ is the identity,
we call $\mcU$ a \emph{refinement} of $\mcV$.
For a $ U \in \mcC$ the coverings together with refinements
gives a category $\Cov(U)$.
This allows the following reformulation:
\begin{lemma}
    \label{lem:presheaf_of_cover}
    Let $\mcF$ be a presheaf on a site $\mcC$.
    For $U \in \mcC$ and a cover $\mcU \in \Cov(U)$, define
    $\mcF(\mcU)$ as the following equalizer:
    \begin{equation*}
        \mcF(\mcU) =
        \{(s_i)_{i\in I} \in \prod\limits_{i\in I}\mcF(U_i)\mid
        s_i|_{U_i \times_U U_j} = s_j|_{U_i \times_U U_j}\}.
    \end{equation*}
    If $\mcU \to \mcV$ is a morphism of coverings, there is an induced
    map $\mcF(\mcV) \to \mcF(\mcU)$. This construction is functorial.
    Furthermore, since $\{1_U\}$ is a cover
    by the axioms of a site, this gives a map $\mcF(U) \cong \mcF(\{1_U\}) \to \mcF(\mcU)$.
    The presheaf $\mcF$ is a sheaf if and only if the map
    $\mcF(U) \to \mcF(\mcU)$ is bijective
    for every cover $\mcU$.
\end{lemma}
\begin{proof}
    Let $\mcU \to \mcV$ be a map of coverings. We have the following
    commutative diagram:
    \begin{equation*}
        \begin{tikzcd}
            & U_i \rar \dar &V_{\alpha(i)} \dar \\
            U_i \times_U U_j \ar[ru] \ar[rd]
            & U \rar & V
            & V_{\alpha(i)} \times_V V_{\alpha(j)} \ar[lu] \ar [ld] \\
            & U_j \rar \uar & V_{\alpha(j)} \uar
        \end{tikzcd},
    \end{equation*}
    which gives a unique map $U_i \times_U U_j \to V_{\alpha(i)} \times_V V_{\alpha(j)}$
    making the diagram commute. Applying $\mcF$ gives the diagram:

    \begin{equation}\label{eq:fibre_product_coverings}
        \begin{tikzcd}
            \mcF(U_i) \dar &\mcF(V_{\alpha(i)}) \lar \dar \\
            \mcF(U_i \times_U U_j) & \mcF(V_{\alpha(i)} \times_V V_{\alpha(j)} )  \lar \\
            \mcF(U_j) \uar &\mcF(V_{\alpha(j)}) \lar \uar
        \end{tikzcd}
        .
    \end{equation}

    We can now define $\mcF(\mcV) \to \mcF(\mcU)$ by
    \begin{equation*}
        (s_j)_{j\in J} \mapsto ((s_{\alpha(i)})|_{U_i})_{i\in I}.
    \end{equation*}
    That this map is well-defined follows from \cref{eq:fibre_product_coverings}.
    Indeed, if $(s_j)_{j\in J} \in \mcF(\mcV)$, then
    \begin{equation*}
        (s_{\alpha(j)}|_{V_{\alpha(i)}\times_U V_{\alpha(j)}})|_{U_i \times U_j}
        = (s_{\alpha(i)}|_{V_{\alpha(i)}\times_U V_{\alpha(j)}})|_{U_i \times U_j},
    \end{equation*}
    and hence
    \begin{equation*}
        (s_{\alpha(i)}|_{U_i})|_{U_i \times_U U_j}
        = (s_{\alpha(j)}|_{U_j})|_{U_i \times_U U_j}.
    \end{equation*}

    It remains to show that this construction is functorial, since
    the final claim is just the definition of a sheaf of sites. Let,
    to this purpose, $\mcU \to \mcV \to \mcW$ be maps of coverings,
    with associated maps $\alpha\colon I \to J$ and $\beta \colon J \to K$.
    The map $\mcF(\mcW) \to \mcF(\mcU)$ maps $(s_k)_{k\in K}$ to
    $(s_{\beta(\alpha(i))}|_{U_i})_{i\in I}$. The other map is given by
    first sending $(s_k)_{k \in K}$ to $(s_{\beta(j)}|_{V_j})_{j\in J}$,
    which is then sent to $((s_{\beta(\alpha(i))}|_{V_{\alpha(i)}})|_{U_i})_{i\in I}$.
    The result now follows by functoriality of $\mcF$.
\end{proof}
We now have the following construction:
\begin{lemma}
    Given a presheaf of $\mcF$ on a site $\mcC$, we can construct
    a new presheaf $\mcF^+$ by setting
    \begin{equation*}
        \mcF^+(U) = \colim_{\mcU \in \Cov(U)} \mcF(\mcU),
    \end{equation*}
\end{lemma}
\begin{proof}
    Let us first show how $\mcF^+$ acts on morphisms. Consider
    a morphism $U \to V$. Then this gives a map $\Cov(V) \to \Cov(U)$
    by mapping a cover $\{V_i \to V\}_{i\in I}$ of $V$ to the cover
    $\{V_i \times_V U \to U\}_{i\in I}$ of $U$,
    which exists by the third axiom of coverings.
    From the previous lemma there are induced maps
    ${\mcF}(\mcU) \to {\mcF}(\mcV)$.
    Since we have maps $\mcF(\mcV) \to \mcF^+(V)$,
    we obtain maps ${\mcF}(\mcU) \to \mcF^+(V)$.
    By the universal property of the colimit, this gives a unique map
    $\mcF^+(V) \to \mcF^+(U)$, making the cone diagrams commute.
    \begin{equation*}
        \begin{tikzcd}[column sep=small]
            &&\mcF^+(U) \ar[ddll, bend right, leftarrow]
            \ar[ddrr, bend left, leftarrow]\\
            &&\mcF^+(V)\uar[dashed]\\
            \mcF(\mcU) \rar &\mcF(\mcV)\ar[ur] \ar[rr]
            &&\mcF(\mcV') \ar[ul] &\mcF(\mcU') \lar
        \end{tikzcd}
    \end{equation*}

    It remains to show that this defines a functor $\mcC^{\opp}\to \Set$.
    If we have morphisms $U\to V \to W$, then
    \begin{equation*}
        W_i \times_W U \cong (W_i \times_W V) \times_V U,
    \end{equation*}
    since both are pullbacks of $\begin{tikzcd}
            [sep=small, cramped]
            W_i \rar & W & U \lar
        \end{tikzcd}$.
    So $\Cov(-)$ is a functor, and
    since $\mcU \mapsto \mcF(\mcU)$ is a functor
    by the previous lemma, the result now follows.
\end{proof}
It is possible to be more explicit about how $\mcF^+$ looks like.
For two coverings $\mcU, \mcU'$ of $U \in \mcC$,
we have a common refinement $\{U_i \times_U U'_j \to U\}_{i\in I, j\in J}$
which exists by the second and third axioms.
Furthermore, one can show that if $f,g\colon \mcU \to \mcV$
are refinements, then $\mcF(f) = \mcF(g)$
(\cite[\href{https://stacks.math.columbia.edu/tag/00W7}{Tag 00W7}]{stacks-project}).
This gives that $\Cov(U)^{\opp} \to \Set$ is a filtered
diagram. So
\begin{equation*}
    \mcF^+(U) =\Bigl( \coprod\limits_{\mcU \in \Cov(U)} \mcF(\mcU)\Bigr)/ \sim,
\end{equation*}
Where $s \sim s'$ if and only if there are covers $\mcU, \mcU'$ with
$s\in \mcF(\mcU), s' \in \mcF(\mcU')$ and a common refinement $\mcV$ such that
\begin{equation*}
    s_{\alpha(i)}|_{V_i} = s'_{\beta(i)}|_{V_i}, \, \forall i\in I.
\end{equation*}
We now come to the main theorem, from which
\cref{prop:sheafification} will follow.
\begin{theorem}
    Let $\mcF$ be a presheaf on a site. Then
    \begin{enumerate}
        \item The presheaf $\mcF^+$ is separated.
        \item If $\mcF$ is separated, then $\mcF^+$ is a sheaf.
        \item If $\mcF$ is a sheaf, then $\mcF \to \mcF^+$ is an isomorphism.
    \end{enumerate}
\end{theorem}
\begin{proof}
    \leavevmode
    \begin{enumerate}
        \item We need to show that $s \mapsto (s|_{U_i})_{i\in I}$ is injective
              for any cover $\{U_i \to U\}_{i \in I} \in \Cov(\mcC)$.
              Take $s, s' \in \mcF^+(U)$ such that
              $(s|_{U_i})_{i\in I} = (s'|_{U_i})_{i\in I}$ for some
              cover $\mcU = \{U_i \to U\}_{i\in I}$.
              By the description above of $\mcF^+$, we know that
              we can find covers $\mcV$ and $\mcV'$ of $U$,
              such that $s\in \mcF(\mcV)/\sim$ and $s'\in\mcF (\mcV')/\sim$.
              Let $\mcW$ be a common refinement of the three covers
              $\mcU, \mcV, \mcV'$. Since it is a refinement of $\mcU$,
              we have that
              \begin{equation*}
                  s'|_{W_j} = (s|_{U_{\alpha(j)}})|_{W_j} =
                  (s|_{U_{\alpha(j)}})|_{W_j} = s'|_{W_j},
              \end{equation*}
              so that $s=s'$.
        \item We need to verify the sheaf condition. Let $\{U_i \to U\}_{i\in I}$
              be a cover of $U$, and for each $i\in I$, $s_i \in \mcF^+(U_i)$ such that
              $s_i |_{U_i \times_U U_j} = s_j |_ {U_i \times_U U_j}$ for every $i,j \in I$.
              Since $\mcF$ is separated, the map $s \to (s|{U_i})_{i\in I}$ is injective.
              It is hence enough to show that there is some $s\in \mcF^+(U)$ with
              $s|_{U_i} = s_i$ for all $i\in I$.
              For each $i\in I$ we can find a cover $\mcU_i = \{U_{ij} \to U_i\}$
              such that $s_i \in \mcF(\mcU_i)/\sim$, and hence $s_{ij} \in \mcF(U_{ij})$
              such that $s_i |_{U_{ij}} = s_{ij}/\sim$. In the same way
              as \cref{lem:presheaf_of_cover} we get that
              \begin{equation*}
                  s_{ij}|_{U_{ij}\times_U U_{i'j'}} = s_{i'j'}|_{U_{ij}\times_U U_{i'j'}}.
              \end{equation*}
              So, $(s_{ij})_{i,j\in I} \in \mcF(\{U_{ij}\to U\}_{i,j\in I})$, and we can
              take $s = (s_{ij})_{i,j\in I}/\sim$.
              We just need to verify that $s|_{U_i} = s_i$. This follows
              from $(s|_{U_i})|_{U_{ij}} = s|_{U_{ij}} = s_i|_{U_{ij}}$, as $\mcF^+$ is
              also separated.
        \item This follows from \cref{lem:presheaf_of_cover}.
    \end{enumerate}
\end{proof}

With this we are now ready to prove \cref{prop:sheafification}.
We define $\mcF^\sharp = \mcF^{++}$.
\begin{proof}[Proof of \cref{prop:sheafification}]
    We first note that for any map of presheaves $\alpha\colon \mcF \to \mcG$,
    we have a commutative diagram
    \begin{equation*}
        \begin{tikzcd}
            \mcF \rar \dar["\alpha"] & \mcF^+ \dar["\alpha^+"] \\
            \mcG \rar & \mcG^+
        \end{tikzcd},
    \end{equation*}
    where $\alpha^+_U(s/ \sim) = \alpha_U(s)/\sim$.
    That this commutes, is by construction of $\alpha^+$.
    Using this, we can now show that
    \begin{equation*}
        \Hom_{\Sh(\mcC)}(\mcF^\sharp, \mcG) = \Hom_{\PSh(\mcC)}(\mcF, \iota(\mcG)).
    \end{equation*}
    The bottom row of the diagram
    \begin{equation*}
        \begin{tikzcd}
            \mcF \rar \dar &\mcF^+ \dar \rar & \mcF^{++} = \mcF^{\sharp} \dar \\
            \iota(\mcG) \rar  &\iota(\mcG)^+  \rar & \iota(\mcG)^{++} = \mcG^{\sharp}
        \end{tikzcd}
        ,
    \end{equation*}
    consists of isomorphisms by the previous theorem.
    So, any map $\mcF \to \iota(\mcG)$ gives rise to a map
    $\mcF^\sharp \to \mcG$.
    Conversely, since every $s\in \mcF^\sharp(U)$ comes from sections
    in $\mcF(U)$, we can lift any map $\mcF^\sharp \to \mcG$
    to a map $\mcF \to \mcG$.
\end{proof}

As an immediate consequence, we get that $(-)^\sharp$
commutes with all colimits. So $\Sh(\mcC)$ has all colimits,
since $\PSh(\mcC)$ has all colimits by \cref{lem:lims_colims_presheaves}.
We can say more:
\begin{prop}
    \label{prop:sheafification_exact}
    The functor $(-)^\sharp\colon \PSh(\mcC) \to \Sh(\mcC)$ is exact.
\end{prop}
\begin{proof}
    Since it is a left adjoint, it is right exact. On the other hand,
    colimits over filtered diagrams commute with finite limits.
    So, $(-)^\sharp$ is left exact as functor between presheaves.
    We claim that if $\mcI \to \Sh(\mcC)$ is a diagram, then
    the limit $\mcF = \lim_i \mcF_i$ as presheaves is a sheaf.
    For this we show that $\mcF(\mcU) \cong \mcF(U)$ for any cover $\mcU = \{U_j \to U\}$.
    Take $(s_j)_{j\in J} \in \mcF(\mcU)$, then by definition of the
    limit, we can project these to elements $(s_{ij})_{j \in J} \in \mcF_i(\mcU)$.
    Since each $\mcF_i$ is a sheaf, we have unique elements $s_i \in \mcF_i(U)$
    such that $s_i |_{U_j} = s_{ij}$.

    We now want an element $s\in \mcF(U)$ with projections equal to the $s_i$.
    Choosing an element of $\mcF(U)$, is the same as giving a map $\{*\} \to \mcF(U)$,
    which by the universal property of the limit is the same as giving a cone
    $(\{*\}, \lambda_i)$. Let $\lambda_i(*) = s_i$, then we just need to verify
    that this defines a cone.
    We need that for $f\colon i\to i'$ in $\mcI$,
    $\mcF(f)(s_i) = s_{i'}$. This follows, as $s_i|_{U_j}$ is mapped to $s_{i'}|_{U_j}$
    for all $j\in J$, and hence $s_i$ is mapped to $s_{i'}$ because
    $\mcF_{i'}$ is a sheaf. So we have a unique $s\in \mcF(U)$,
    and by the universal property of the limit
    \begin{equation*}
        s|_{U_j} = s_j \iff s_i|_{U_j} = s_{ij}\; \forall i\in I,
    \end{equation*}
    which holds by construction.
    Hence, the claim holds, and $(-)^\sharp$ is also right exact as a functor
    into sheaves.
\end{proof}
To get a better understanding of sheafification, let us work out
the process of sheafification for some presheaves.
\begin{example}
    We consider the presheaf $\mcF$ of \cref{exmp:bad_presheaves}.
    To calculate $\mcF^+$ we use that
    \begin{equation*}
        \mcF^+(U) = \Bigl(\coprod\limits_{\mcU \in \Cov(U)}\mcF(\mcU)\Bigr)/ \sim.
    \end{equation*}
    If $U$ is non-empty, then
    \begin{equation*}
        \mcF(\mcU) = \{(y_i)_{i\in I}\in \prod_{i\in I} Y \mid y_i = y_j\; \forall i,j \in I\}
        = \{(y)_{i\in I}\mid y\in Y\}
    \end{equation*}
    for any cover $\mcU$ of $U$. Take $(y)_{i\in I} \in \mcF(\mcU)$, and
    $(y')_{j\in J} \in \mcF(\mcU')$. For any common refinement $\mcV$ of
    $\mcU$ and $\mcU'$, we have $(y_{\alpha(i)})|_{V_i} = y$ and $(y'_{\beta(i)})|_{V_i} = y'$
    so that $(y)_{i\in I} \sim (y')_{j\in J} \iff y = y'$.
    So, we find again that $\mcF^+(U) = Y$ as long as $U \neq \emptyset$.

    Now, when $U = \emptyset$ there are two covers: $\{\emptyset \to \emptyset\}$,
    and the empty covering, $\{\}_{i\in \emptyset}$. We have
    $\mcF(\{\emptyset \to \emptyset\}) = \mcF(\emptyset) = Y$,
    while $\mcF(\{\}_{i\in \emptyset}) = \{*\}$.
    Now every $y\in Y$ is equivalent to $*$, as $\{\}_{i\in \emptyset}$
    is a common refinement of the two covers, and the equivalence
    condition becomes an empty statement in this case. As such we find
    $\mcF^+(\emptyset) = \{*\}$.
    Note that $\mcF^+$ is isomorphic to $\mcG$ from \cref{exmp:bad_presheaves}.
    After all, a constant function $U \to Y$ is the same
    as choosing an element in $Y$.

    We now have a separated presheaf $\mcF^+$. What does the sheaf
    $\mcF^\sharp$ look like? Let us first consider the case that $U$
    is empty. Then for both covers $\{\emptyset \to \emptyset\}$ and
    $\{\}_{i\in \emptyset}$ we get $\mcF^+(\mcU) = \{*\}$.
    So $\mcF^\sharp(\emptyset) = \{*\}$.

    More interesting is what happens when $U \neq \emptyset$.
    Let $\{U_i \to U\}_{i\in I}$ be a cover such that
    none of the $U_i$ are empty. Then
    \begin{align*}
        \mcF^+(\mcU) & =
        \{(y_i)_{i\in I}\in \prod_{i\in I} Y \mid
        y_i|_{U_i \cap U_j} = y_j|_{U_i \cap U_j}\; \forall i,j \in I\} \\
                     & = \{(y_i)_{i\in I}\in \prod_{i\in I}Y\mid
        y_i = y_j,\, U_i \cap U_j \neq \emptyset\},
    \end{align*}
    as $\mcF^+(\emptyset) = \{*\}$ and hence $y|_{U_i \cap U_j} = *$
    if $U_i \cap U_j = \emptyset$ for any $y \in Y$.
    We can view elements of $\mcF^+(\mcU)$ as functions $s\colon U \to Y$,
    that are constant on each of the opens $U_i$.
    Adding empty sets
    to the cover does not really change anything: the elements of
    $\mcF(\mcU)$ will be $*$ at the indices for which $U_i = \emptyset$.
    Under $\sim$ such functions will all be the same.
    Take, $f \in \mcF^\sharp(U)$. Then there is some cover
    $\mcU = \{U_i\}_{i\in I}$ such that $f$
    arises from $\mcF^+(\mcU)$. Consequently, for each $x\in U$,
    there is some $U_i$ containing $x$, such that $f$
    is constant on $U_i$. In other words, $f$ is a locally
    constant function on $U$. If we equip $Y$ with the discrete
    topology then this is equivalent to saying that $f\colon U \to Y$
    is continuous.

    The sheaf $\mcF^\sharp$ is (perhaps a little confusingly) called the
    \emph{constant sheaf with value $Y$} and sometimes
    denoted as $\underline{Y}$.
\end{example}
\section{Sheaf cohomology}

In the previous section we have explored some
fundamental properties of sheaves of sets.
We will now be looking at sheaves of abelian groups,
which form a very interesting and rich abelian category.
The previous section only proved results for sheaves of
sets, but these results still hold for sheaves of abelian
groups. This can be shown by either verifying that all
the proofs still hold in the abelian case, or through
a more abstract approach like in
\cite[\href{https://stacks.math.columbia.edu/tag/00YR}{Section 00YR}]{stacks-project}
\begin{prop}
    Let $\mcC$ be a site. The category
    $\Ab(\mcC)$ of abelian sheaves on $\mcC$ is abelian.
\end{prop}
\begin{proof}
    Let $\mcF$ and $\mcG$ be two abelian sheaves on $\mcC$.
    For natural transformations $\alpha, \beta \in \Hom_{\Ab(\mcC)}(\mcF, \mcG)$
    we define $\alpha + \beta$ via
    \begin{equation*}
        (\alpha + \beta)_U \defeq \alpha_U + \beta_U,
    \end{equation*}
    where the $+$ on the right-hand side is the $+$ in $\Ab$.
    Let us verify that $\alpha + \beta$ is a natural transformation.
    If $f\colon U \to V$ is a morphism in $\mcC$ we have:
    \begin{align*}
        (\alpha + \beta)_U \circ \mcF(f)
         & = (\alpha_U + \beta_U) \circ \mcF(f)             \\
         & = \alpha_U \circ \mcF(f) + \beta_U \circ \mcF(f) \\
         & = \mcG(f) \circ \alpha_V + \mcG(f) \circ \beta_V \\
         & = \mcG(f) \circ (\alpha_V + \beta_V)             \\
         & = \mcG(f) \circ (\alpha + \beta)_V,
    \end{align*}
    so that $\alpha + \beta$ is indeed an abelian category.
    With this operation, the hom-sets become abelian groups.
    Using \cref{lem:lims_colims_presheaves} and \cref{prop:sheafification_exact}
    we find that $\Ab(\mcC)$ has finite limits and colimits.
    Explicitly, limits in sheaves are the limits in presheaves
    and colimits in sheaves are the sheafifications of the colimits in presheaves.
    As such, we find:
    \begin{itemize}
        \item The zero object is given by $\underline{0}$,
              the constant sheaf with value $0$.
        \item The biproduct of $\mcF$ and $\mcG$ is given by
              $(\mcF \oplus \mcG )(U) = \mcF(U) \oplus \mcG(U)$
        \item The kernel of a map $\alpha \colon \mcF \to \mcG$ is given by
              $\Ker(\alpha)(U) = \Ker(\alpha_U)$.
        \item The cokernel of $\alpha \colon \mcF \to \mcG$
              is given by the sheafification of the presheaf defined by
              $\mcF(\alpha)(U) = \Coker(\alpha_U)$.
    \end{itemize}
    For $\Ab(\mcC)$ to be an abelian category, we still need to
    show that for $\alpha \colon \mcF \to \mcG$ a morphism of sheaves,
    $\Img(\alpha) = \Coimg(\alpha)$. In the category of abelian presheaves we have
    \begin{equation*}
        \Coimg(\iota(\alpha))(U) = \Coimg(\iota(\alpha)_U)
        = \Img(\iota(\alpha)_U) = \Img(\iota(\alpha))(U),
    \end{equation*}
    where $\iota$ is the inclusion functor of sheaves into presheaves.
    Since $(-)^\sharp$ is exact, we find
    \begin{align*}
        \Coimg(\alpha)
         & = \Coker(\Ker(\alpha))               \\
         & = \Coker^\sharp(\iota(\Ker(\alpha))) \\
         & = \Coker^\sharp(\Ker(\iota(\alpha))) \\
         & = \Coimg^\sharp(\iota(\alpha))       \\
         & = \Img^\sharp(\iota(\alpha))         \\
         & = \Ker^\sharp(\Coker(\iota(\alpha))) \\
         & = \Ker(\Coker^\sharp(\iota(\alpha))) \\
         & = \Ker(\Coker(\alpha))               \\
         & = \Img(\alpha).
    \end{align*}
\end{proof}

Taking the sheafification is really needed when
taking colimits. This is illustrated in the following
example.
\begin{example}
    Consider the two sheaves $\Cont(-, \bbR)$ and
    $\Cont(-, 2\pi \bbZ)$ on the circle $S^1 \subseteq \bbC$.
    We claim that the quotient $\Cont(-, \bbR)/ \Cont(-, 2\pi \bbZ)$
    as presheaves is not a sheaf. For this, consider
    the two open sets
    $U_1 = \{e^{i\theta} \in S^1 \mid \theta \in(0, 2\pi) \}$ and
    $U_2 = \{e^{i\theta} \in S^1 \mid \theta \in(-\varepsilon, \varepsilon) \}$.
    We have continuous functions $f_i \colon U_i \to \bbR$ given by
    $f_1(e^{i\theta}) = \theta \in (0, 2\pi )$, and
    $f_2(e^{i\theta}) = \theta \in (- \varepsilon, \varepsilon)$.
    Let $V = U_1 \cap U_2 = \{e^{i\theta} \in S^1 \mid \theta \in (-\varepsilon,0)\cup(0, \varepsilon) \}$.
    Then
    \begin{equation*}
        (f_1 |_V - f_2 |_V)(e^{i \theta}) =
        \begin{cases}
            2 \pi & \theta \in (-\varepsilon, 0) \\
            0     & \theta \in (0, \varepsilon)  \\
        \end{cases},
    \end{equation*}
    so $f_1|_{V} - f_2|_V \in \Cont(V, 2 \pi \bbZ)$ as it is locally constant.
    Consequently, the two functions agree on overlaps in the quotient.
    If $\Cont(-, \bbR)/\Cont(-, 2\pi \bbZ)$
    were a sheaf, then we would be able to glue the two functions together
    to a function $\bar{f} \in \Cont(S^1, \bbR)/ \Cont(S^1 , 2\pi \bbZ)$
    such that $\bar{f_1} = \bar{f}|_{U_1}$ and $\bar{f_2} = \bar{f}|_{U_2}$.
    Since $S^1$ is connected, the elements of $\Cont(S^1, 2 \pi \bbZ)$ are
    constant functions. So, $f_1 |_V - f|_V$, and $f_2 |_ V - f|_V$ would
    have to be constants, and hence also $f_1 |_V - f_2 |_V$, a contradiction.
\end{example}

We will now restrict ourselves to the special case
where $\mcC = \Open(X)$ for some topological space $X$.
We follow the general ideas presented in \cite[Session 1]{Sch2020MasterClass},
while filling out the details.
\begin{prop}
    Let $f \colon X \to Y$ be a continuous map. Then there is a
    pair of adjoint functors: the pullback functor,
    $f^* \colon Ab(Y) \to Ab(X)$, with right adjoint, the pushforward functor
    $f_* \colon Ab(X) \to Ab(Y)$. Furthermore, $f^*$ is exact.
\end{prop}
\begin{proof}
    We start off by writing down what the pullback and pushforward
    functors look like.
    Let $\mcF$ be a sheaf on $X$, and $V \in \Open(Y)$. Then,
    $(f_*\mcF)(V) \defeq \mcF(f\inv(V))$.
    If $\mcG$ is a sheaf on $Y$, then
    we get a presheaf $V \mapsto \colim_{U \supseteq f(V)} \mcG(U)$,
    and $f^*\mcG$ is defined as the sheafification of this presheaf.
    There are now a lot of things to verify:
    \begin{itemize}
        \item $f_*\mcF$ is a sheaf on $Y$. Since $f$ is continuous,
              $f\inv(V) \in \Open(X)$ for all $V\in \Open(Y)$. Furthermore,
              if $V \subseteq V' \subseteq V''$ then
              $f\inv(V) \subseteq f\inv(V') \subseteq f\inv(V'')$. So,
              $f\inv$ is a functor $\Open(Y) \to \Open(X)$, and $f_*\mcF$
              is a presheaf as composition of functors. If $\{V_i\}_{i\in I}$
              is an open cover of $V$, then $\{f\inv(V_i)\}_{i\in I}$
              is an open cover of $f\inv(V)$.
              So $(f_*\mcF)(V) \to (f_*\mcF)(\{V_i\}_{i\in I})$ is a bijection
              since $\mcF(f\inv(V)) \to \mcF(\{f\inv(V_i)\}_{i\in I})$
              is a bijection by $\mcF$ being a sheaf. Consequently,
              $f_*\mcF$ is a sheaf by \cref{lem:presheaf_of_cover}.
        \item $f^*\mcG$ is a sheaf on $X$. Since we take the sheafification,
              we just need to verify that $G(U) = \colim_{V \supseteq f(U)} \mcG(V)$
              defines a presheaf. First note that if $U\subseteq U'$,
              then $\{V \supseteq f(U')\} \subseteq \{V \supseteq f(U)\}$.
              So, $G(U)$ is also a co-cone for $\mcG(V)_{V \supseteq f(U')}$,
              and there is hence a unique map $G(U') \to G(U)$,
              by the colimit property. The uniqueness
              of this map gives the functoriality of $G$. After all,
              the map $G(U) \to G(U)$ must be the identity, by uniqueness,
              and the two maps $G(U'') \to G(U') \to G(U)$, for
              $U \subseteq U' \subseteq U''$ must also be
              equal by uniqueness.
        \item $f_*$ is a functor. For $\alpha \colon \mcF \to \mcG$,
              we define $f_*\alpha$ via $(f_*\alpha)_V = \alpha_{f\inv(V)}$.
              Then,
              \begin{equation*}
                  (f_*\beta \circ f_* \alpha)_V
                  = \beta_{f\inv(V)} \circ \alpha_{f\inv(V)}
                  = (\beta \circ \alpha)_{f\inv(V)}
                  = (f_*(\beta\circ \alpha))_V,
              \end{equation*}
              so we just need to check that $f_*\alpha$ is a natural
              transformation. If $V \subseteq V'$, then $f\inv(V) \subseteq f\inv(V')$,
              so naturality of $f_*\alpha$ in $V,V'$ follows from naturality
              of $\alpha$ in $f\inv(V), f\inv(V')$.
        \item $f^*$ is a functor. Let $\alpha\colon \mcG \to \mcF$
              be a natural transformation, and $G$ and $F$ be the presheaves
              from the construction of $f^*\mcG$ and $f^*\mcF$. To define
              $f^*\alpha$ we need to give a map $G(U) \to F(U)$ for each
              $U\in \Open(X)$. For each $V \supseteq f(U)$ we have a
              map $\mcF(V) \to F(U)$ which we can compose with $\alpha_V$
              to get a map $\mcG(V) \to F(U)$. Naturality of $\alpha$
              implies that this gives a well-defined co-cone, and hence a
              unique morphism $G(U) \to F(U)$ by the universal property
              of colimits. Uniqueness of this map gives the functoriality
              of $f^*$.
        \item $\Hom_{\Ab(X)}(f^*\mcG, \mcF) = \Hom_{\Ab(Y)}(\mcG, f_*\mcF)$.
              Let $\alpha \colon \mcG \to f_*\mcF$ be a morphism in $\Ab(Y)$.
              So, for each $V\in \Open(Y)$ we have a map
              \begin{equation*}
                  \alpha_V \colon \mcG(V) \to \mcF(f\inv(V)).
              \end{equation*}
              Now, if $V \supseteq f(U)$, then $f\inv(V) \supseteq f\inv(f(U)) \supseteq U$.
              So we get a map
              \begin{equation*}
                  \beta_U^V \defeq \alpha_V|_U \colon \mcG(V) \to\mcF(f\inv(V))\to \mcF(U),
              \end{equation*}
              for each $V \supseteq f(U)$. In order for $(\mcF(U), \beta_U^V)$ to
              form a co-cone we need that
              \begin{equation*}
                  \begin{tikzcd}[column sep=small, cramped]
                      &\mcF(U)\\
                      \mcF(f\inv(V)) \ar[rr, "|_{f\inv(V')}"] \ar[ru, "|_U"]
                      & & \mcF(f\inv(V')) \ar[lu, "|_U"'] \\
                      \mcG(V) \ar[rr, "|_{V'}"] \ar[u, "\alpha_V"] & &
                      \mcG(V') \ar[u, "\alpha_{V'}"']
                  \end{tikzcd}
                  ,
              \end{equation*}
              commutes for $V \supseteq V' \supseteq f(U)$.
              The top triangle commutes by functoriality of $\mcF$,
              while the bottom square commutes by naturality of $\alpha$.
              The universal property of the colimit gives a unique map
              \begin{equation*}
                  \tilde{\beta}_U \colon \colim_{V \supseteq f(U)} \mcG(V) \to \mcF(U),
              \end{equation*}
              making the co-cone diagrams commute.
              We now need to verify that this gives a map of presheaves
              $\tilde{\beta} \colon G \to \mcF$,
              where $G$ is the presheaf in the construction
              of $f^*\mcG$. Since
              \begin{equation*}
                  \begin{tikzcd}[sep=small]
                      G(U)\dar \rar["\tilde{\beta}_U"] & \mcF(U) \dar\\
                      G(U') \rar["\tilde{\beta}_{U'}"] & \mcF(U')
                  \end{tikzcd}
                  \iff
                  \begin{tikzcd}[sep=small]
                      \mcG(V) \rar["\alpha_V"] \dar
                      & \mcF(f\inv(V)) \rar \dar
                      & \mcF(U) \dar \\
                      \mcG(V) \rar["\alpha_{V'}"]
                      & \mcF(f\inv(V)) \rar
                      & \mcF(U)
                  \end{tikzcd}
                  , \forall V \supseteq f(U), V' \supseteq f(U'),
              \end{equation*}
              this is indeed the case. After all, the right-hand side
              commutes by naturality of $\alpha$ and functoriality of $\mcF$.
              Finally, applying sheafification
              gives the map $\beta \colon f^*\mcG \to \mcF$ we needed.

              Conversely, given $\beta \colon f^*\mcG \to \mcF$, we have maps
              \begin{equation*}
                  \beta_U^V \colon \mcG(V) \to \mcF(U)
              \end{equation*}
              for every $U \in \Open(X)$ and $V \supseteq f(U)$.
              Since $V \supseteq f(f\inv(V))$, we can define
              \begin{equation*}
                  \alpha_V \defeq \beta_{f\inv(V)}^V \colon \mcG(V) \to \mcF(f\inv(V)).
              \end{equation*}
              By naturality of $\beta$,
              \begin{equation*}
                  \begin{tikzcd}[cramped]
                      f^*\mcG(f\inv(V)) \dar["\beta_{f\inv(V)}"'] \rar
                      & f^*\mcG(f\inv(V')) \dar["\beta_{f\inv(V')}"]\\
                      \mcF(f\inv(V))  \rar & \mcF(f\inv(V'))
                  \end{tikzcd}
              \end{equation*}
              commutes, and hence in particular we get
              \begin{equation*}
                  \begin{tikzcd}[cramped]
                      \mcG(V) \dar["\beta^{V}_{f\inv(V)}"'] \rar
                      & \mcG(V') \dar["\beta^{V'}_{f\inv(V')}"]\\
                      \mcF(f\inv(V))  \rar & \mcF(f\inv(V'))
                  \end{tikzcd}
              \end{equation*}
              which shows the naturality of $\alpha$.

              We now need to check that the two constructions are inverses
              of each other. Starting from $\alpha \colon \mcG \to f_* \mcF$
              we obtain $\beta \colon f^*\mcG \to \mcF$.
              Let $\gamma\colon \mcG \to f_* \mcF$ be the map we obtain
              from $\beta$. We want to show that $\gamma = \alpha$.
              Take $V \in \Open(X)$, then
              \begin{equation*}
                  \gamma_V = \beta_{f\inv(V)}^V = \alpha_V|_{f\inv(V)} = \alpha_V,
              \end{equation*}
              as desired. Conversely, given $\beta \colon f^*\mcG \to \mcF$,
              we obtain $\alpha \colon \mcG \to f_* \mcF$ from $\beta$, and
              $\eta \colon f^*\mcG \to \mcF$ from $\alpha$. We have,
              \begin{equation*}
                  \eta_U^V = \alpha_V |_U = \beta_{f\inv(V)}^V |_U,
              \end{equation*}
              for $V \supseteq f(U)$. Since $U \subseteq f\inv(V)$,
              naturality of $\beta$ gives $\beta_{f\inv(V)}^V |_U = \beta_U^V$,
              so $\eta_U^V = \beta_U^V$.

              The final thing to check is that the transformations are natural
              in $\mcF$ and $\mcG$. Let us denote $\Phi$ for the map
              from $\Hom_{\Ab(X)}(f^* \mcG, \mcF)$ to $\Hom_{\Ab(Y)}(\mcG, f_*\mcF)$.
              Let us start with naturality in $\mcF$.
              If $\gamma \colon \mcF \to \mcF'$ is a natural transformation, we need
              to show that $\Phi(\gamma \beta) = f_*\gamma \Phi(\beta)$ for any
              morphism $\beta \colon f^*\mcG \to \mcF$. Indeed,
              \begin{equation*}
                  \Phi(\gamma \alpha)_V = (\gamma \alpha)_{f\inv(V)}^V
                  = \gamma_{f\inv(V)}\alpha_{f\inv(V)}^V
                  = (f_*\gamma)_V \Phi(\alpha)_V.
              \end{equation*}
              For $\eta \colon \mcG \to \mcG'$, we need to show that
              $\Phi(\beta f^*\eta) = \Phi(\beta)\eta$. The following
              calculation shows that this is indeed the case
              \begin{equation*}
                  \Phi(\beta f^* \gamma)_V =(\beta f^* \gamma)_{f\inv(V)}^V
                  = \beta_{f\inv(V)}^V \gamma_V = \Phi(\beta)_V \gamma_V.
              \end{equation*}


        \item $f^*$ is exact. Since it is a left adjoint, it is right exact.
              We still need to show that it is left exact, i.e. $f^*$ preserves
              monomorphisms. Let $\alpha\colon \mcF \to \mcG$ be a monomorphism.
              Since $\Ker(\alpha)(V) = \Ker(\alpha_V)$ we have that $\alpha$
              is mono if and only if all the $\alpha_V$ are mono. Since
              colimits over filtered diagrams commute with finite limits,
              it follows that $F(U) \to G(U)$ is a mono if and only
              if all the $\alpha_V$ for $V \supseteq f(U)$ are mono.
              Consequently, $f^*\alpha$ is a mono if $\alpha$ is a mono.
    \end{itemize}
\end{proof}
This result allows us to prove the crucial property
needed to be able to do cohomology in $\Ab(X)$.
\begin{lemma}
    $\Ab(X)$ has enough injectives.
\end{lemma}
\begin{proof}
    Take $x\in X$. Then $i_{x}\colon \{x\} \injto X$ is
    a continuous map and hence induces
    the pushforward functor $i_{x,*} \colon \Ab(\{x\}) \to \Ab(X)$.
    Note that $\Ab(\{x\})$ is equivalent to $\Ab$ as category,
    since a sheaf on one point is completely determined
    by the value at that point.
    Now if $M$ is an injective object in $\Ab$, we claim that $i_{x,*}M$ is injective
    in $\Ab(X)$. Let $\alpha \colon \mcF \to \mcG$ be a monomorphism and
    $\beta\colon \mcF \to i_{x, *}M$ any morphism of sheaves. We need to
    find a map $\gamma \colon \mcG \to i_{x, *}M$ such that $\beta = \gamma\circ \alpha$.
    Since $i_x^*$ is a left adjoint to $i_{x, *}$ we get a map
    $b \colon i_x^*(\mcF) \to M$. Since $i_x^*$ is exact, the map
    $a \colon i_x^*\mcF \to i_x^*\mcG$ is still a monomorphism. So,
    since $M$ is injective, there is a map $g\colon i_x^*\mcG \to M$
    such that $b = g\circ a$. This gives a map $\gamma \colon \mcG \to i_{x, *}M$,
    and $\gamma \circ \alpha$ is the same as the induced map by $g\circ a = b$
    since an adjunction is natural in both arguments.
    Thus, $\gamma \circ \alpha = \beta$ as desired.

    What remains to show is that for every sheaf $\mcF$, there is a monomorphism
    $\mcF \injto i_{x, *}M$, for some $x\in X$ and injective $M\in \Ab$.
    Since $\Ab$ has enough injectives, there is a monomorphism $a_x\colon i_x^*\mcF \to M_x$,
    for every sheaf $\mcF$. The adjunction gives a map $\alpha_x\colon \mcF \to i_{x,*}M_x$.
    The problem is that this map need not be a monomorphism,
    since $(i_{x, *}M)(U) = M(i_x\inv(U)) = \{*\}$ if $x\notin U$.
    Since the product of injectives is again injective, we can solve this
    problem by considering the map
    \begin{equation*}
        \alpha = (\alpha_x)_{x\in X} \colon \mcF \to \prod\limits_{x\in X}i_{x,*}M,
    \end{equation*}
    instead. Take $V \in \Open(X)$. If $V = \emptyset$, then $\mcF(V) = \{*\}$
    and $\alpha_V$ is a monomorphism. Otherwise, take $x\in V$. We claim
    that $(\alpha_x)_V \colon \mcF(V) \to M_x$ is a monomorphism. By definition,
    $(\alpha_x)_V$ is given by the composition of $\mcF(V) \to i_x^*\mcF(V)$
    and $i_x^*\mcF(V) \to M_x$. Since filtered diagrams commute with
    finite limits, we have that $(\alpha_x)_V$ is indeed a monomorphism.
    So, if $x,y\in V$ with $\alpha_V(x) = \alpha_V(y)$, then $x=y$,
    which shows that $\alpha$ is a monomorphism.
\end{proof}

In the previous proof we considered the inclusion map.
If we instead consider the projection map $f\colon X \to \{*\}$,
then the pullback $f^*$ becomes the map $\Ab \to \Ab(X)$,
which sends an abelian group $M$ to the constant sheaf with
value $M$. The pushforward $f_*\colon \Ab(X) \to Ab$ maps
a sheaf, $\mcF$, to its global sections, $\mcF(X)$.

With this we can now do cohomology in $\Ab(X)$.
Let $f\colon X \to Y$ be a continuous function.
Since $f_*$ is a right adjoint, it is left exact, and
we can consider the right derived functors $R^if_* \colon \Ab(X) \to \Ab(Y)$.

\begin{example}
    Let us look at the specific example where $f \colon S^1 \to \{*\}$
    is the projection map. We have an exact sequence of sheaves on $S^1$
    \begin{equation*}
        \begin{tikzcd}
            0 \rar & \Cont(-, 2\pi \bbZ) \rar["f"] & \Cont(-, \bbR)
            \rar["g"] & \Cont(-, S^1) \rar & 0.
        \end{tikzcd}
    \end{equation*}
    To check that $f$ is a monomorphism, we can check it locally. Since
    each of the maps $f_U \colon \Cont(U, 2\pi \bbZ) \to \Cont(U, \bbR)$ are injective,
    so is $f$. The kernel of $g$ can be computed locally as well.
    For every open $U$ in $S^1$, we have that $h \in \Cont(U, \bbR)$ is
    mapped to $\pi \circ h$ where $\pi \colon \bbR \to S^1 = \bbR / (2\pi \bbZ)$
    is the quotient map. So the kernel of $g_U$ is given by the functions
    whose image lies in $2\pi \bbZ$. In other words, $\Ker(g_U) = \Cont(U, 2\pi \bbZ)$.
    This shows exactness at the middle. What remains to show is that $g$ is an
    epimorphism. For this, we need to check that for every open $U$ in $S^1$,
    and continuous function $s \in \Cont(U, S^1)$, there is a cover ${U_i}_{i\in I}$
    of $U$ such that $s|_{U_i}$ is in the image of $g_U$. Let $U_1 = U \cap S^1 \setminus \{1\} $
    and $U_2 = U \cap S^2 \setminus \{2\}$. Then $\{U_1,U_2\}$ is an open cover
    of $U$, and each of the elements in the cover is homeomorphic to an open subset of $\bbR$.
    So for $s\in \Cont(U, S^1)$, we have that $s|_{U_1}$ and $s|_{U_2}$
    are in the image of $g_{U_1}$ and $g_{U_2}$.

    On the other hand, the map $\Cont(S^1 , \bbR) \to \Cont(S^1, S^1)$ is
    not an epimorphism. After all, $\bbR$ is contractible, so any
    function in $\Cont(S^1, \bbR)$ is homotopic to the zero function. Hence,
    all the functions in the image will also be homotopic to the zero map.
    Since $S^1$ has a non-trivial fundamental group, we see that image
    is not everything.

    Consequently, we only get a left exact sequence
    \begin{equation*}
        \begin{tikzcd}
            0 \rar & \Cont(S^1, 2\pi \bbZ) \rar["f"] & \Cont(S^1, \bbR)
            \rar["g"] & \Cont(S^1, S^1),
        \end{tikzcd}
    \end{equation*}
    when we apply $f_*$. It follows that the derived functor $R^1f_*$ is
    non-zero. In fact, one can show that $R^1f_*(\Cont(-, \bbZ)) = \bbZ$.
    More generally, $R^if_*(\Cont(-, \bbZ)) = H^i(S^1 , \bbZ)$, where
    $H^i(S^1, \bbZ)$ is the singular cohomology.
\end{example}

As in the example, we define the sheaf
cohomology of a topological space $X$ with coefficients in an
abelian group $A$ as
\begin{equation*}
    H^i_{\text{sheaf}}(X, A) \defeq R^if_*(\Cont(-, A)),
\end{equation*}
where $A$ is given the discrete topology and $f$ is the
projection map $f\colon X \to \{*\}$.
\section{Condensed sets}

We now have all the necessary material to be able to say a little
about condensed sets. Intuitively, a condensed set
is a generalization of a topological space, which contains
more information about how it interacts with other spaces.
The goal of this section is two-fold:
\begin{itemize}
    \item We want to define what a condensed set is,
          and compute some examples.
    \item We want to define cohomology in the condensed
          setting and compare it to other types of cohomology.
\end{itemize}
\subsection{What is a condensed set?}
As it turns out, the definition of a condensed set is not so
straightforward. As is the case in representation theory,
it is often useful to study an object by considering
representations of the object as simpler objects. In
this case we will study topological spaces by considering
maps from profinite sets into topological spaces.
\begin{definition}[Profinite set]
    A profinite set is a compact, Hausdorff, totally disconnected space.
    Explicitly, it is a topological space $X$, such that
    \begin{itemize}
        \item Every open cover of $X$ has a finite subcover.
        \item For every two different points, $x\neq y \in X$, there
              exist neighborhoods $U$ and $V$ of $x$ and $y$ such that $U\cap V = \emptyset$.
        \item The connected components of $X$ are the singletons.
    \end{itemize}
\end{definition}
For example, a finite space with the discrete topology is a profinite set.
We would now like to turn the category of profinite sets into
a site. There is a small problem though. Namely, the category
of profinite sets isn't small. To solve this, we
take an uncountable strong limit cardinal $\kappa$
and only consider the profinite sets with cardinality
less than $\kappa$. Recall that a strong limit
cardinal is a cardinal $\kappa$ such that
$2^\lambda < \kappa$ for all cardinals $\lambda < \kappa$.

What are the coverings on this category? A covering
is given by a finite family of jointly surjective
maps, thus a set of maps $\{f_i \colon X_i \to X\}_{i=0}^n$
such that $X = \bigcup_{i=0}^n f_i(X_i) = X$.
This collection of coverings does indeed satisfy the needed axioms:
\begin{enumerate}
    \item If $f\colon Y \to X$ is an isomorphism, then $f$ is surjective,
          so $\{f\}$ is a cover.
    \item If $\{f_i \colon X_i \to Y\}_{i=0}^n$ is jointly surjective, and
          for each $i \in \{0, \dots, n\}$, the set $\{f_{ij}\colon X_{ij} \to X_i \}_{j=0}^{n_i}$
          is jointly surjective, then so is the composition
          $\{f_i \circ f_{ij}\colon X_{ij} \to Y\}_{i,j=0}^{n, n_i}$.
    \item Finally, let $\{f_i \colon X_i \to Y\}_{i= 0}^n$ be a cover, and
          $f\colon S \to Y$ a morphism. Then, assuming that the pullback
          $X_i \times_Y S$ is a profinite set, we have that the
          projection maps $\{X_i \times_Y S \to S\}_{i=0}^n$ are jointly surjective.
          After all, for $s\in S$, the image $f(s)$ is equal to $f_i(x_i)$ for some
          $i \in \{0, \dots, n\}$ since the $f_i$ are jointly surjective.
          So, $s$ is the image of $(x_i, s) \in X_i \times_Y S$.
\end{enumerate}

The only thing left to show is that the pullback of two profinite
sets is again a profinite set. In fact, something stronger
is true:
\begin{prop}
    \label{prop:limit_profinite}
    A limit of profinite spaces is again a profinite space.
\end{prop}
\begin{proof}
    Let $X = \lim_{i \in \mcI} X_i$ be the limit as topological spaces.
    We first show that $X$ is still Hausdorff and totally disconnected.
    Take $x\neq y \in X$. Then for at least one of the projections
    $f_i \colon X \to X_i$ we must have $f_i(x) \neq f_i(y)$. Since
    $X_i$ is Hausdorff, we can find open neighborhoods $U$ of
    $f_i(x)$ and $V$ of $f_i(y)$ such that $U\cap V = \emptyset$.
    But then $f_i\inv(U) \cap f_i\inv(V) = \emptyset$, and
    $x \in f_i\inv(U), y \in f_i\inv(V)$. Since $X_i$ is totally
    disconnected, $f_i(x)$ and $f_i(y)$ are in separate components.
    Since the continuous image of a connected set is connected,
    we must have that $x$ and $y$ are in separate
    connected components as well.
    Finally, one can also show that limits of compact Hausdorff
    spaces are again compact (see
    \cite[\href{https://stacks.math.columbia.edu/tag/08ZV}{Lemma 08ZV}]{stacks-project}).
    Essentially the argument is as follows:
    \begin{itemize}
        \item Limits are made out of products and equalizers.
        \item Products of compact spaces are compact.
        \item By above, limits of Hausdorff spaces are Hausdorff.
        \item Equalizers of functions into Hausdorff spaces are closed.
        \item So the limit is compact as a closed subspace of a compact space.
    \end{itemize}
\end{proof}
With this we arrive at the following definition:
\begin{definition}[$\kappa$-condensed sets]
    The category of $\kappa$-condensed sets is the
    category of sheaves on the site of $\kappa$-small
    profinite sets, with coverings given by
    jointly surjective maps.
\end{definition}
Analogously, we can define condensed abelian groups, rings, vector spaces\dots.
As we have seen, the category of abelian sheaves on a site is abelian.
This is not true for the category of topological abelian groups. After all,
in any abelian group, a morphism that is both an epimorphism and a
monomorphism is automatically an isomorphism. The map
\begin{equation*}
    \id\colon (\bbR, \mcT_{\text{discrete}}) \to (\bbR, \mcT_{\text{Euclidean}}),
\end{equation*}
is bijective as sets, and hence an epimorphism and a monomorphism. However,
since the two topologies don't coincide, the map is not an isomorphism.

How is this issue resolved in the condensed setting?
With a topological space $X$, we can associate a
condensed set $\underline{X} = \Cont(-, X)$. The
``underlying set'' can then be found as $\underline{X}(\{*\})$.
The point is that the map
\begin{equation*}
    \underline{\id}\colon \underline{\bbR_{\text{discrete}}} \to \underline{\bbR_{\text{Euclidian}}}
\end{equation*}
is no longer an epimorphism. As we will see later on, the cokernel
of this map can be computed ``section-wise'', i.e. $\Coker(f)(U) = \Coker(f_U)$.
So, to show that the cokernel is not trivial, we need to show that
there is some profinite set $S$ with
$\Cont(S, \bbR_{\text{discrete}}) \subsetneq \Cont(S,\bbR_{\text{Euclidean}})$.
Consider the Cantor set $S$, which is homeomorphic to $\prod_{n\in \bbN}\{0,1\}$
where $\{0,1\}$ has the discrete topology. By \cref{prop:limit_profinite}, the
Cantor set is a profinite set. It can alternatively be described as the
closed subspace of the real numbers (with the Euclidean topology), given by
\begin{equation*}
    S = [0,1]\setminus \bigcup_{n=0}^\infty \bigcup_{k=0}^{3^n-1}
    \left(\frac{3k+1}{3^{n+1}},\frac{3k+2}{3^{n+1}}\right).
\end{equation*}
With this description, we get a continuous map from $S$ into the real numbers
with the Euclidean topology, namely the inclusion map. On the other hand,
this map is not continuous when $\bbR$ is equipped with the discrete topology.
If it were, then every singleton in $S$ would have to be open, as the inverse
image of an open set. Since $S$ doesn't have the discrete topology, this is
not the case. Consequently, the map $f_S$ is not an epimorphism.

\medskip
Recall from \cref{lem:lims_colims_presheaves} that limits and colimits
of presheaves can be taken section-wise. For sheaves, the colimit is
not always a sheaf, and we may need to take the sheafification of the
colimit as presheaves to obtain a sheaf. For the category of
$\kappa$-condensed sets, this is no longer necessary. To
prove this, we'll need an alternative description of condensed sets.
\begin{definition}
    An \emph{extremally disconnected} set is a projective object
    in the category of compact Hausdorff spaces\footnote{
        The usual definition is that the closure of
        every open set must again be open. In this setting
        the two definitions are equivalent
        (\cite[Theorem 2.5]{Gle1958ProjectiveTS})}.
\end{definition}
One way to construct these extremally disconnected spaces, is
through Stone--\v{C}ech compactification.
\begin{theorem}[Stone--\v{C}ech compactification]
    Let $X$ be a topological space. There is a unique (up to isomorphism)
    compact Hausdorff space $\beta X$ together with a continuous map
    $i_X \colon X \to \beta X$, such that for any continuous map
    $f \colon X \to K$, with $K$ a compact and Hausdorff space, there is
    a unique extension of $f$ to a continuous map $\beta f \colon \beta X \to K$:
    \begin{equation*}
        \begin{tikzcd}
            X \rar["i_X"] \ar[rd, "f"'] & \beta X  \dar[dashed, "\beta f"]\\
            & K
        \end{tikzcd}
    \end{equation*}
    In terms of cardinality, we have $|\beta X| \leq 2^{\left(2^{|X|}\right)}$.
\end{theorem}
See, for example, \cite[\href{https://stacks.math.columbia.edu/tag/0908}{Section 0908}]{stacks-project}
for a proof. Now, let $X$ be a compact Hausdorff space,
and let $X'$ be the same space equipped with the discrete topology.
Then the map $\id \colon X' \to X$ is continuous, and by
the universal property of the Stone--\v{C}ech compactification,
we have a factorization:
\begin{equation*}
    \begin{tikzcd}
        X' \rar["i_{X'}"] \ar[rd, "\id"'] & \beta X ' \dar[dashed, "\beta \id"]\\
        & X
    \end{tikzcd}.
\end{equation*}
Since $\id$ is bijective, it follows
that $i_{X'}$ is injective, and $\beta \id$ is surjective.

We claim that $\beta X'$ is extremally disconnected.
So, let $e \colon K \surjto S$ be an epimorphism between compact Hausdorff spaces,
and $f\colon \beta X' \to S$ be any map.
Since any set is projective in $\Set$, there is a map (of sets)
$g \colon X' \to K$ such that $e\circ g = f \circ i_{X'}$.
Since $X'$ has the discrete topology, the map $g$ is continuous.
Hence, by the universal property of the Stone--\v{C}ech compactification,
there is a unique continuous map $\beta g \colon \beta X' \to K$ such that
$g = \beta g \circ i_{X'}$.
\begin{equation*}
    \begin{tikzcd}
        X' \rar[hook, "i_{X'}"] \dar["g"] & \beta X \dar["f"] \ar[ld, dashed, "\beta g"] \\
        K \rar[two heads, "e"'] & S
    \end{tikzcd}
\end{equation*}
So
\begin{equation*}
    f \circ i_{X'} = e \circ g = e \circ \beta g \circ i_{X'},
\end{equation*}
and hence $f = e \circ \beta g$, since $i_{X'}$ is a monomorphism.

So, we have shown that any compact Hausdorff space admits a surjection from
and extremally disconnected space, i.e. the category has enough projectives.
We now want to describe $\kappa$-condensed sets, by restricting to
just the category of extremally disconnected spaces. However, there is a small
technicality that we have to take care of first. The product of two
infinite extremally disconnected spaces is never extremally disconnected. So,
we can't take fiber products, and so there is a problem in defining a site
on this category. The solution is to consider an alternative description
of what a site is.

One defines a \emph{sieve} on an object $X \in \mcC$ as a subfunctor
of its Yoneda embedding $\Hom(- ,X)$. So, it is a functor $S \colon \mcC^{\opp} \to \mcC$
such that for any $X' \in \mcC$ we have $S(X') \subseteq \Hom(X', X)$,
and for all morphisms $f\colon Y \to Y'$, $S(f)$ is the restriction
of $\Hom(f, X)$ to $S(Y')$. A site becomes a category together with
a Grothendieck topology. Namely, for each object of $\mcC$,
we choose a collection of sieves on that object, called covering sieves.
These must again satisfy axioms similar to those for covers:
\begin{enumerate}
    \item $\Hom(-, X)$ is a covering sieve for any $X$ in $\mcC$.
    \item If $S$ is a covering sieve on $X$, $T$ is sieve on $X$,
          and for any $f\in S$ the sieve $f^*T$ is a covering sieve,
          then $T$ is a covering sieve.
    \item If $S$ is a covering sieve on $X$, and $f\colon X \to Y$ is
          a morphism, then the pullback $f^*S$ is a covering sieve on $Y$.
\end{enumerate}
Here the pullback of $S$ by $f\colon Y \to X$ is given by
$f^*S(Z)= \{g\colon Z \to Y \mid f\circ g \in S(Z)\}$.
Importantly, the definition doesn't require the existence of pullbacks.
The sheaf condition for a presheaf $\mcF$, is that for any
object $X$ and any covering sieve $S$ on $X$, the map
\begin{equation*}
    \Hom(\Hom(-, X), \mcF) \to \Hom(S, \mcF)
\end{equation*}
induced by the inclusion of $S$ in $\Hom(-, X)$, is a bijection.
In other words, every natural transformation from $S$ to $\mcF$
extends uniquely to a natural transformation from $\Hom(-, X)$
to $\mcF$.

A collection of covers satisfying the axioms from our original definition
of a site, is referred to as a pretopology. Given such a pretopology
we can make a Grothendieck topology by taking as covering sieves,
precisely those sieves which contain a cover from the pretopology.
The crucial part is that these two sites give rise to the same category
of sheaves (\cite[Corollary 1.1.28]{Dag2021FoundationsCM}).

So we can define the site of $\kappa$-small extremally
disconnected spaces, with covers given by finite families
of jointly surjective maps. That this generates a Grothendieck
topology is shown in \cite[Proposition 1.2.12]{Dag2021FoundationsCM}.
Finally, it turns out that restricting to $\kappa$-small extremally disconnected
spaces gives an equivalent category of sheaves, to the category
of $\kappa$-condensed sets (\cite[Theorem 1.2.16]{Dag2021FoundationsCM}).
Here it is crucial that the cardinality of the Stone--\v{C}ech compactification
is bounded. Since $\kappa$ is an uncountable strong limit cardinal,
we know that if $|X| < \kappa$ then also $|\beta X| < \kappa$.

Checking that a presheaf is a sheaf, becomes a lot easier now.
A sheaf on $\kappa$-small
extremally disconnected spaces is a functor
\begin{equation*}
    T \colon \{\text{extremally disconnected spaces}\}^{\opp} \to \Set / \Rng / \Ab/ \dots
\end{equation*}
with $T(\emptyset) = \{*\}$, and
such that $T(S_1 \sqcup S_2) \to T(S_1) \times T(S_2)$ is a bijection
for any two extremally disconnected sets $S_1, S_2$
with cardinality smaller than $\kappa$ (\cite[Theorem 1.2.18]{Dag2021FoundationsCM}).
As a result we find:
\begin{corollary}
    The category of $\kappa$-condensed sets is abelian. Limits
    and colimits can be taken section-wise.
\end{corollary}
\begin{proof}
    Note that limits and colimits in $\Ab$ commute with finite products.
    Additionally, we already saw that the section-wise limit and colimit
    of presheaves is again a presheaf (\cref{lem:lims_colims_presheaves}).
    Now assume that $(T_i)_{i\in \mcI}$ are sheaves. Taking their
    limit or colimit as presheaves will then again be a sheaf,
    since
    \begin{align*}
        (\lim_{i\in \mcI}T_i)(S_1 \sqcup S_2)
         & = \lim_{i\in \mcI} T_i(S_1 \sqcup S_2)                            \\
         & = \lim_{i\in \mcI} (T_i(S_1) \times T_i(S_2))                     \\
         & = \lim_{i\in \mcI} T_i(S_1) \times \lim_{i\in \mcI} T_i(S_2)      \\
         & = (\lim_{i\in \mcI} T_i)(S_1) \times (\lim_{i\in \mcI} T_i)(S_2),
    \end{align*}
    and similarly for colimits.
\end{proof}

So, problems with colimits, where we need to take the sheafification
do not occur in this setting. As a final thing to consider, let
us take a look at cohomology in the condensed setting.

\subsection{Condensed cohomology}

Given all the work that we have done in constructing cohomology
for arbitrary categories of sheaves on a topological space,
it would not be unreasonable to assume that we would now use this
to define cohomology of condensed abelian groups.
Although we won't be able to copy over the construction exactly,
it will serve as our source of inspiration. There are a few
problems that we need to address first.

Ideally, we would want the definition of condensed sets
to be independent of an (arbitrary) choice of uncountable
limit cardinal. This is possible, and is done
in \cite[Appendix to Lecture II]{Sch2019LecturesCM}. The resulting
category is still abelian, and limits and colimits
can still be taken section-wise. However, this category
is no longer the category of sheaves on a site, and so some
properties might no longer be valid. One important one
is that the category no longer has enough injectives. In fact,
it has no non-zero injectives (\cite{Sch2020AreTE}). So,
we will need to use projective resolutions instead.
Let us show that there are enough projectives in
the category of $\kappa$-condensed abelian groups.
\begin{lemma}
    The forgetful functor $U$ from $\kappa$-condensed
    abelian groups has a left adjoint $\bbZ[-]$.
    For extremally disconnected $S$, $\bbZ[\underline{S}]$
    is projective.
\end{lemma}
\begin{proof}
    Let $T$ be a $\kappa$-condensed set, and $\bbZ[T]$ be the sheafification of the presheaf
    which sends $S$ to $\bbZ[T(S)]$, the free abelian group on $T(S)$.
    To see that this is a left adjoint, note that for each extremally disconnected
    space $S$, and condensed abelian group $M$, we have a natural isomorphism
    \begin{equation*}
        \phi_{T, M, S}\colon \Hom_{\Ab}(\bbZ[T(S)], M(S)) \cong \Hom_{\Set}(T(S), U(M)(S)),
    \end{equation*}
    since free abelian groups are a left adjoint to the forgetful functor.
    This then give the desired isomorphism
    \begin{equation*}
        \phi_{T, M} \colon \Hom_{\Cond(\Ab)}(\bbZ[T], M) \cong \Hom_{\Cond(\Set)}(T, U(M)),
    \end{equation*}

    Since limits and colimits can be taken section-wise, i.e. $M \mapsto M(S)$
    commutes with all limits and colimits, it follows that $\bbZ[\underline{S}]$
    is projective. Indeed, in any abelian category, an object $P$ is
    projective if and only if $\Hom(P, -)$ is exact. So, to check that $\bbZ[\underline{S}]$
    is projective we need that $\Hom_{\Cond(\Ab)}(\bbZ[\underline{S}], -)$ is exact.
    This is equivalent to saying that the map $M \mapsto M(S)$
    is exact, since $\Hom_{\Cond(\Ab)}(\bbZ[\underline{S}], M) = M(S)$ for any
    condensed abelian group $M$, by the Yoneda lemma. Since $M \mapsto M(S)$
    commutes with limits, it is indeed exact.
\end{proof}
This lemma and the following corollary are still valid in the actual definition of condensed sets,
which is independent on a choice of cardinal $\kappa$.
\begin{corollary}
    The category of $\kappa$-condensed abelian groups has enough projectives.
\end{corollary}
\begin{proof}
    We follow the proof given in \cite[Theorem 2.2.6]{Dag2021FoundationsCM}.
    Let $M$ be a $\kappa$-condensed abelian group. For any extremally
    disconnected set $S$, and $x\in M(S)$, we have a morphism
    $\alpha_x \colon \bbZ[\underline{S}] \to M$ such that $\alpha_x(\id_{S}) = x$,
    by the adjunction and the Yoneda lemma. Then the induced map
    \begin{equation*}
        h \colon \bigoplus_S \bigoplus_{x\in M(S)}\bbZ[\underline{S}] \to M,
    \end{equation*}
    is surjective. Since all the $\bbZ[\underline{S}]$ are projective,
    so is the direct sum, and hence we have found a surjection onto $M$
    from a projective object.
\end{proof}

\medskip
Recall that sheaf cohomology was defined via right-derived
functors of $\mcF \mapsto \mcF(X)$. We want to define
condensed cohomology as derived functors of $M \mapsto M(S)$.
Since we don't have injective resolutions, we can't take
the right derived functors the usual way. However, the
Yoneda lemma and the above adjunction give us a way around it.
We have
\begin{equation*}
    \Hom_{\Cond(\Ab)}(\bbZ[\underline{S}], M) \cong
    \Hom_{\Cond(\Set)}(\underline{S}, M) \cong
    M(S),
\end{equation*}
and hence we can use the $\Ext$ functors instead.
Concretely, for a topological space $X$ and abelian group $A$,
the condensed cohomology of $X$ with coefficients in $A$ is
\begin{equation*}
    H^i_{\text{condensed}}(X, A) \defeq \Ext^i(\bbZ[\underline{X}], \underline{A}).
\end{equation*}

It should come as no surprise that this agrees with sheaf cohomology
for nice topological spaces $X$. We have the following results:
\begin{theorem}[ \protect{\cite[Theorem 3.2 and 3.3]{Sch2019LecturesCM}} ]
    If $X$ is a compact Hausdorff space, then
    \begin{equation*}
        H^i_{\textrm{condensed}}(X, \bbZ) \cong H^i_{\textrm{sheaf}}(X, \bbZ),
    \end{equation*}
    and
    \begin{equation*}
        H^i_{\textrm{condensed}}(X, \bbR) \cong 0, i>0,\qquad
        H^0_{\textrm{condensed}}(X, \bbR) \cong \Cont(X, \bbR).
    \end{equation*}
\end{theorem}

\nocite{Apa2021condensed}
\bibliographystyle{alpha}
\bibliography{references.bib}
\end{document}
